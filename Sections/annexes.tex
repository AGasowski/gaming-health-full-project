\section*{Annexes}
\addcontentsline{toc}{section}{Annexes}

\begin{center}\includegraphics[width=1\textwidth]{Images/tableau_jeuA_PCSper-1.png}\\[2.0 cm]\end{center}
\textit{Note de lecture: 18,6\% des individus avec un père ouvrier jouent à des jeux d'argent moins d'une foi par mois}
Une autre variable socio-démographique est la catégorie socio-professionnelle des parents.
Elle est d’autant plus importante ici car les sujets sont jeunes et beaucoup dépendent financièrement,
au moins partiellement, de leurs parents. Intéressons-nous ici à la PCS du père.
Lorsqu’on analyse cette variable avec la consommation de jeu d’argent on se rends compte qu’elle 
semble déterminer à grande échelle cette pratique. Nous avons évoqué plus haut l’aspect 
onéreux des jeux d’argent, et cela peut donc 
rejoindre le fait que les enfants de cadre soient ceux qui jouent le plus souvent.