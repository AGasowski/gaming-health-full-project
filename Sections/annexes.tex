\section*{Annexes}
\addcontentsline{toc}{section}{Annexes}

\begin{center}
    \textbf{\Large Annexe A}\label{ann:pcs}\\[1em]
    \includegraphics[width=1\textwidth]{Images/tableau_jeuA_PCSper-1.png}\\[2.0cm]
\end{center}
\textit{Note de lecture: 18,6\% des individus avec un père ouvrier jouent à des jeux d'argent moins d'une foi par mois}
Une autre variable socio-démographique est la catégorie socio-professionnelle
des parents. Elle est d’autant plus importante ici car les sujets sont jeunes
et beaucoup dépendent financièrement, au moins partiellement, de leurs parents.
Intéressons-nous ici à la PCS du père. Lorsqu’on analyse cette variable avec la
consommation de jeu d’argent on se rends compte qu’elle semble déterminer à
grande échelle cette pratique. Nous avons évoqué plus haut l’aspect onéreux des
jeux d’argent, et cela peut donc rejoindre le fait que les enfants de cadre
soient ceux qui jouent le plus souvent.

\vspace{4em}

\begin{center}
    \textbf{\Large Annexe B}\label{ann:reduction}\\[1em]
\end{center}
Afin d'observer des résultats pertinents et d'améliorer la compréhension de
l'étude, nous avons choisi de réduire les modalités de fréquence des variables.
L'organisation de cette réduction est expliqué ci-dessous:

\begin{table}[H]
    \centering
    \captionsetup{justification=centering}
    \begin{tabular}{lccc}
          \hline
          Réponse de l'individu & Nouvelle modalité\\
          \hline
          Jamais & Jamais \\
          \hline
          Une fois par mois ou moins & Moins d'une fois par semaine \\
          \hline
          2-3 fois par mois & Moins d'une fois par semaine \\
          \hline
          Une fois par semaine & Au moins une fois par semaine \\
          \hline
          Plusieurs fois par semaine & Au moins une fois par semaine \\
          \hline
          Tous les jours ou presque & Au quotidien \\

    \end{tabular}
    \vskip 10pt \caption{Réduction des modalités des variables de fréquence de
    jeu}
\end{table}

\vspace{1em}

\begin{table}[H]
    \centering
    \captionsetup{justification=centering}
    \begin{tabular}{lccc}
          \hline
          Réponse de l'individu & Nouvelle modalité\\
          \hline
          Jamais & Jamais ou presque jamais \\
          \hline
          Moins d'une fois par mois & Jamais ou presque jamais \\
          \hline
          Une fois par mois & Jamais ou presque jamais \\
          \hline
          Une fois par semaine & Une fois par semaine \\
          \hline
          2 fois par semaine & Plusieurs fois par semaine \\
          \hline
          3 fois par semaine & Plusieurs fois par semaine \\
          \hline
          4 à 6 fois par semaine & Plusieurs fois par semaine \\
          \hline
          Chaque jour & Au quotidien \\

    \end{tabular}
    \vskip 10pt \caption{Réduction des modalités des variables de fréquence de
    pratique sportive}
\end{table}

\vspace{4em}

\begin{center}
    \textbf{\Large Annexe C}\label{ann:correlation}\\[1em]
    \textbf{Tableau des tests de corrélations}\\[0.5em]
    \includegraphics[width=1\textwidth]{Annexes/Tableau de
    corrélations.png}\\[2.0cm]
\end{center}
Pour réaliser ces calculs, nous avons réalisé les calculs sans les valeurs
manquantes, et avec les modalités réduites des variables comme expliqué dans
l'annexe B.

\vspace{4em}

\begin{center}
    \textbf{\Large Livrable 1 }\label{ann:livrable}\\[1em]
    \textbf{Livrable 1}\\[0.5em]
    \includepdf[pages=-]{Livrable1.pdf}
\end{center}
