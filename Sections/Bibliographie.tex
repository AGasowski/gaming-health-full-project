\section*{Bibliographie}
\addcontentsline{toc}{section}{Bibliographie}

\renewcommand{\refname}{} %enlève le "Références"

\begin{thebibliography}{99}
\normalsize
% Exemple pour la bibliographie  \
% \bibitem[numero]{motclé} où le numéro est le numéro dans la bibliographie et
% le motclé permettra de faire référence au numéro avec la typographie
% \cite{motclé} 

\bibitem[1]{SELL2023} Syndicat des Éditeurs de Logiciels de Loisirs (SELL),
\textit{L'Essentiel du Jeu Vidéo - Octobre 2023}, L'Essentiel du Jeu Vidéo,
Octobre 2023


\bibitem[2]{Eroukmanoff2024} Vincent Eroukmanoff, \textit{Les jeux d'argent et
de hasard en France en 2023}, Observatoire français des drogues et des
tendances addictives (OFDT), Juillet 2024


\bibitem[3]{OFDT2022} Observatoire français des drogues et des tendances
addictives (OFDT), \textit{Drogues et addictions, chiffres clés 2022}, 2022

\bibitem[4]{gambling} Lucy T Tran et al., 
\textit{The prevalence of gambling and problematic gambling: a systematic review and meta-analysis}
The Lancet Public Health Volume 9, 2024

\bibitem[5]{gamer} Samuel Coavoux, 
\textit{La différenciation genrée des pratiques des jeux vidéo.}
Enjeux numériques, 2019, 6, pp.35-38. ffhalshs-02152427f

\bibitem[6]{enfance_loisirs} Berthomier Nathalie et Sylvie Octobre. 
\textit{L’enfance des loisirs}
L’enfance des loisirs, Département des études, de la prospective et des statistiques, 2011, 
https://books.openedition.org/deps/219

\end{thebibliography}
