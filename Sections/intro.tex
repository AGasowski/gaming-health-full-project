\section*{Introduction}
\addcontentsline{toc}{section}{Introduction}

En juin 2023, Emmanuel Macron a le sentiment que les jeux vidéo « ont intoxiqués »
des émeutiers. En février 2025, Bruno Retailleau propose d’expliquer un meurtre par
une « addiction aux jeux vidéo ». Il semble que pour le pouvoir exécutif français, la
consommation de jeux vidéo et la santé mentale sont intimement liées. Ce sont deux
sujets très contemporains qui méritent qu’on s’intéresse à leur relation. Mais
l’association des termes ‘jeux’ et ‘addiction’ peut évoquer un autre genre de jeu :
les jeux d’argents. Il serait alors intéressant de regarder le lien entre la santé
dans sa globalité avec les jeux, qu’ils soient des jeux d’argent ou des jeux vidéo.

\vspace{1cm}

D’après le bilan 2023 de l’OFDT, l’Observatoire français des drogues et
tendances addictives, environ 50\% de la population majeure française a joué à
au moins un jeu d’argent pendant l’année écoulée. De même, près de 70\% de la
population française affirme jouer au moins une fois par an selon une étude
faite par le SELL, Syndicat des éditeurs de logiciels de loisirs. Internet a
joué un grand rôle dans la popularité grandissante de ces deux catégories de
jeux. En effet internet a permis aux joueurs de jouer aux jeux vidéos en ligne
démocratisant leur utilisation. De nouveaux jeux sont constamment développés à
travers le monde pour viser un plus large public. De plus, internet a aussi été
important pour l’industrie des jeux d’argent avec l’apparition des casinos en
ligne et des sites de paris sportifs. Ces derniers permettent aux joueurs de
continuer à jouer de leur domicile. L’accès aux jeux d’argent et aux jeux
vidéos est donc grandement facilité et les joueurs peuvent se livrer sans
limite à cette manière de se divertir.

\vspace{1cm}

Néanmoins, la pratique de ces jeux soulève également des problématiques,
notamment l’impact négatif de leur consommation excessive sur les joueurs. En
effet, les jeux d’argent ainsi que les jeux vidéo influencent le comportement
des individus, notamment en stimulant le système de récompense du cerveau. Ils
peuvent entraîner divers effets, tels que la sédentarité, des troubles du
sommeil ou encore de l’anxiété. Les joueurs addicts peuvent voir
progressivement leur situation familiale et professionnelle se dégrader
entraînant une potentielle dépression chez l’individu. Les jeux d’argent sont
aussi propices à l’endettement et donc à des difficultés financières pour
certains. En particulier, les jeunes sont grandement exposés aux jeux d'argent
et aux jeux vidéo. Ces derniers sont en constante recherche de nouvelle
expérience et de stimulation qui est facilitée par l’accès accrue à ces jeux.
D'après l’OFDT environ 25\% des jeunes de 17 ans ont déjà joué au jeux d’argent
en 2023 et surtout plus de 90\% des jeunes de 10 à 17 ans jouent régulièrement
aux jeux vidéos selon le SELL.

\vspace{1cm}

Cette situation soulève une interrogation essentielle : dans quelle mesure
existe-il une relation entre les comportements addictifs liés aux jeux et la
santé? Cette question sera au cœur de notre étude, qui portera sur une
population jeune. Nous décrirons d’abord les caractéristiques des individus
avant d'explorer les liens entre la consommation de jeux et la santé physique,
puis la santé mentale.

Notre jeu de données provient d'un questionnaire auto-administré réalisé en
2022 lors de la Journée Défense et Citoyenneté (JDC), portant sur la santé et
les comportements. Ce questionnaire se divise en 3 modules et une partie
commune. L’étude a donc été réduite aux réponses du module B concernant la
consommation de jeux des jeunes.Les questions sur la consommation ne s'appliquent
qu'aux 12 mois précédant la JDC. 
Le jeu de données ne permets pas de diagnostiquer l'addiction au sens 
de l'Organisation mondiale de la santé (OMS).
Nous entendons donc ici par addiction une consommation de jeu très fréquente et répétée.
