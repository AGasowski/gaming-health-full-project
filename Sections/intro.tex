\addcontentsline{toc}{section}{Introduction}

\section*{Introduction}


Présents depuis plusieurs décennies dans les sociétés du monde entier, les marchés des jeux d’argent et des jeux vidéo connaissent une expansion sans précédent, les plaçant parmi les industries les plus influentes. En France, le jeu vidéo demeure le loisir numérique dominant, avec 38,3 millions de pratiquants âgés de 10 ans et plus, représentant 70 % de la population. James Rebours, président du SELL (Syndicat des Éditeurs de Logiciels et de Loisirs), souligne : “2024 démontre une stabilisation à un niveau très élevé du nombre de joueurs réguliers au sein de la population française”. De nouveaux jeux sont constamment développés dans le monde pour viser une audience de plus en plus large et accroitre le nombre de joueurs ainsi que le revenu généré. 

De même, au premier semestre 2024, le Produit Brut des Jeux (PBJ) global du secteur (hors casinos et clubs de jeux) a progressé de 3,8 % par rapport à la même période en 2023, atteignant 5,5 milliards d’euros, porté par les bonnes performances de La Française des Jeux (FDJ) et des opérateurs de jeux en ligne. Internet joue un rôle central dans ce développement avec l’essor des casinos en ligne et des sites de paris. Ce phénomène accroit également la dépendance de certaines personnes aux jeux car ils n’ont désormais plus besoin de se rendre au casino mais peuvent miser et parier en ligne de manière continue. 

Néanmoins, cette expansion soulève également des problématiques, notamment l’impact négatif de la consommation excessive de jeux sur les joueurs. En effet, les jeux d’argent ainsi que les jeux vidéo influencent le comportement des individus, notamment en stimulant le système de récompense du cerveau. Ils peuvent entraîner divers effets, tels que la sédentarité, des troubles du sommeil ou encore de l’anxiété. Les joueurs addicts peuvent voir progressivement leur situation familiale et professionnelle se dégrader entrainant une potentielle dépression chez l’individu. Les jeux d’argent sont aussi propices à l’endettement et donc à des difficultés financières pour certain. En particulier, les jeunes sont grandement exposés aux jeux d’argents et aux jeux vidéo étant la cible de publicité. De plus les jeunes sont en constante recherche de nouvelle expérience et de stimulation qui est facilité par l’accès accrue à ces jeux. 

Nous nous demandons alors : Quel est l'impact des jeux d'argent et des jeux vidéo sur la santé des individus, en particulier les risques liés à l'addiction ? Cette question sera au cœur de notre étude, qui portera sur une population jeune. Dans un premier temps, nous décrirons les caractéristiques des individus avant d'explorer les liens entre la consommation de jeux et la santé physique, puis la santé mentale. 

Notre jeu de données provient d'un questionnaire auto-administré réalisé en 2022 lors de la Journée Défense et Citoyenneté (JDC), portant sur la santé et les comportements de consommation. Nous nous intéressons ici à une population de 17 ans en France métropolitaine. Le questionnaire comporte une section commune ainsi que trois modules distincts, chacun abordant un sujet spécifique. Pour notre analyse, nous avons retenu les réponses du module B, qui est consacré à la santé. 