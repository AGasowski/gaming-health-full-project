\section*{Introduction}
\addcontentsline{toc}{section}{Introduction}

En juin 2023, Emmanuel Macron évoque a le sentiment que les jeux vidéo « ont intoxiqués »
des émeutiers (Le Point, 2023 \cite{le point}). En février 2025, Bruno Retailleau
propose d’expliquer un meurtre par une « addiction aux jeux vidéo » (L'Express, 
2025 \cite{express}). Il semble que pour le pouvoir exécutif français, la
consommation de jeux vidéo et la santé mentale sont intimement liées. Ce sont deux
sujets très contemporains qui méritent qu’on s’intéresse à leur relation. Mais
l’association des termes ‘jeux’ et ‘addiction’ peut évoquer un autre genre de jeu :
les jeux d’argents. Il serait alors intéressant de regarder le lien entre la santé
dans sa globalité avec les jeux, qu’ils soient des jeux d’argent ou des jeux vidéo.

\vspace{1cm}

Les Français consomment beaucoup de jeux. 51,6\% de des français majeurs ont joué
au moins une fois à un jeu d’argent en 2023 (OFDT, 2025) \cite{OFDT2025}. De même, près de 70\% de la
population française affirme jouer au moins une fois par an selon une étude
faite par le SELL, Syndicat des éditeurs de logiciels de loisirs. La popularité de ces
peut se traduire par leur performances économiques ces dernières années. En 2023 et 2024,
le marché des jeux vidéo français a connu les deux meilleures années de son histoire
(SELL, 2025 \cite{SELL2025}). Le marché des jeux d'argent se développe aussi. Le
marché français n'a pas cessé de croître depuis 2020. Les jeux d’argent en ligne 
representent 18,5\% du marché (Autorité Nationale des Jeux, 2025 \cite{ANJ2025})
\vspace{1cm}

Les pouvoirs publics ont vu en ces consommations de potentiels effets nefastes sur la
santé publique. Sur le plan internationnal, l’Organisation Mondiale de la Santé (OMS)
propose instaure une définition de l'addiction, et a déclaré en 2022 le trouble du jeu vidéo
(gaming disorder).
Environ 27,5\% des jeunes de 17 ans ont joué à des jeux d’argent
en 2023 (OFDT, 2025) \cite{OFDT2025}. En 2024, c'est plus de 82\% des 
jeunes de 15 à 24 ans qui jouent
au moins toutes les semaines à des jeux vidéos (SELL, 2024 \cite{SELL2024}).

\vspace{1cm}

Cette situation soulève une interrogation essentielle : dans quelle mesure
existe-il une relation entre les comportements addictifs liés aux jeux et la
santé? Cette question sera au cœur de notre étude, qui portera sur une
population jeune. Nous décrirons d’abord les caractéristiques des individus
avant d'explorer les liens entre la consommation de jeux et la santé physique,
puis la santé mentale.

Notre jeu de données provient d'un questionnaire auto-administré réalisé en
2022 lors de la Journée Défense et Citoyenneté (JDC), portant sur la santé et
les comportements. Ce questionnaire se divise en 3 modules et une partie
commune. L’étude a donc été réduite aux réponses du module B concernant la
consommation de jeux des jeunes.Les questions sur la consommation ne s'appliquent
qu'aux 12 mois précédant la JDC. 
Le jeu de données ne permets pas de diagnostiquer l'addiction au sens 
de l'Organisation mondiale de la santé (OMS).
Nous entendons donc ici par addiction une consommation de jeu très fréquente et répétée.
