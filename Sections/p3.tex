\section{Troisième partie}

\subsection{ADRS et Etat de santé auto déclaré}

La troisième partie de ce rapport se concentre sur l'analyse des liens entre la pratique des jeux et la santé mentale. La santé mentale est un aspect crucial du bien-être général, influençant non seulement la qualité de vie mais aussi la capacité à fonctionner dans divers aspects de la société. Les jeux, en tant qu'activité récréative, peuvent avoir des impacts variés sur la santé mentale, allant de bénéfices potentiels à des risques significatifs.

Pour évaluer l'impact des jeux sur la santé mentale, plusieurs variables et indicateurs ont été utilisés dans cette étude. Parmi eux, l'Adolescent Depression Rating Scale (ADRS) est un outil crucial. L'ADRS est ici une échelle de mesure allant de 0 à 10, permettant de calculer le risque d'Épisode Dépressif Caractérisé (EDC). Cette échelle est représentée dans le graphique \ref{fig:adrs}, illustrant la distribution des risques de dépression parmi les participants.

\begin{figure}[h]
    \centering
    \includegraphics[width=0.8\textwidth]{Images/scoreADRSdistrib.pdf}
    \caption{Proportions de chaque score ADRS.}
    \label{fig:adrs}
\end{figure}

Les graphiques suivants montrent l'état de santé auto-déclaré des joueurs en fonction de leur fréquence de jeu, que ce soit pour les jeux d'argent ou les jeux vidéo. Ces graphiques catégorisent les joueurs selon quatre niveaux d'état de santé : pas du tout satisfaisant, peu satisfaisant, plutôt satisfaisant, et très satisfaisant. Ces catégories permettent d'évaluer comment la fréquence de jeu influence la perception de la santé mentale des individus.

\begin{figure}[h]
    \centering
    \includegraphics[width=0.8\textwidth]{Images/Etatsante_JV.png}
    \caption{État de santé auto-déclaré en fonction de la fréquence de jeux vidéo.}
    \label{fig:jeux_video}
\end{figure}

\begin{figure}[h]
    \centering
    \includegraphics[width=0.8\textwidth]{Images/Etatsante_JA.png}
    \caption{État de santé auto-déclaré en fonction de la fréquence de jeux d'argent.}
    \label{fig:jeux_argent}
\end{figure}

\begin{itemize}
    \item \textbf{Proportions de chaque score ADRS :} Le graphique \ref{fig:adrs} montre la répartition des participants selon leur risque de dépression. Une proportion significative des participants présente un faible risque de dépression (score ADRS bas), tandis qu'une minorité se situe dans les catégories de risque modéré à important. Cette distribution met en lumière la prévalence des risques dépressifs dans la population étudiée.

    \item \textbf{État de santé et jeux d'argent :} Le graphique \ref{fig:jeux_argent} révèle que les joueurs fréquents de jeux d'argent tendent à déclarer un état de santé moins satisfaisant par rapport à ceux qui jouent moins fréquemment. Cette corrélation suggère un lien potentiel entre la fréquence des jeux d'argent et une perception négative de la santé mentale.

    \item \textbf{État de santé et jeux vidéo :} De manière similaire, le graphique \ref{fig:jeux_video} montre que les joueurs fréquents de jeux vidéo ont également tendance à déclarer un état de santé moins satisfaisant. Cependant, la relation semble moins prononcée que pour les jeux d'argent, indiquant que les jeux vidéo pourraient avoir un impact différent ou moins sévère sur la santé mentale perçue.
\end{itemize}


\newpage

\subsection{Deuxième sous-partie}






\newpage

\subsection{Troisième sous-partie}
