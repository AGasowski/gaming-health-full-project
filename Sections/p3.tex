\section{Liens entre la santé mentale et les jeux}

\subsection{Indicateur de l'état de santé mental: le score ADRS}

Après nous être interessés au lien entre la consommation de jeux et la santé
physique, nous nous penchons maintenant sur le lien avec la santé mentale.
Cette dernière constitue en effet un aspect crucial du bien-être chez une
personne.

Pour évaluer les liens entre les jeux et la santé mentale, plusieurs variables
et indicateurs ont été utilisés dans cette étude comme l'Adolescent Depression
Rating Scale (ADRS) et l'ADRS, une échelle de mesure allant de 0 à 10
permettant de calculer le risque d'Épisode Dépressif Caractérisé (EDC). Cette
échelle est représentée dans le graphique \ref{fig:adrs}, illustrant la
distribution des risques de dépression parmi les participants.

\begin{figure}[H]
    \centering
    \includegraphics[width=0.8\textwidth]{Images/scoreADRSdistrib.pdf}
    \caption{Proportions de chaque score ADRS.}
    \label{fig:adrs}
\end{figure}

\begin{itemize}
    \item \textbf{Proportions de chaque score ADRS :} Le graphique
    \ref{fig:adrs} montre la répartition des participants selon leur risque de
    dépression. Une proportion significative des participants (68\%) présente
    un risque de dépression faible (score ADRS < 4), tandis qu'une part non
    négligeable (10\% des jeunes) est soumis à un risque d'EDC fort (score ADRS
    >= 7)

    \item \textbf{État de santé et jeux d'argent :} Le graphique
    \ref{fig:jeux_argent} révèle que les joueurs fréquents de jeux d'argent
    tendent à déclarer un état de santé moins satisfaisant par rapport à ceux
    qui jouent moins fréquemment. Cette corrélation suggère un lien potentiel
    entre la fréquence des jeux d'argent et une perception négative de la santé
    mentale.

    \item \textbf{État de santé et jeux vidéo :} De manière similaire, le
    graphique \ref{fig:jeux_video} montre que les joueurs fréquents de jeux
    vidéo ont également tendance à déclarer un état de santé moins
    satisfaisant. Cependant, la relation semble moins prononcée que pour les
    jeux d'argent, indiquant que les jeux vidéo pourraient avoir un impact
    différent ou moins sévère sur la santé mentale perçue.
\end{itemize}



\subsection{Pratique du jeu vidéo prioritaire et santé mentale}

Cette sous-section explore la relation entre la pratique du jeu vidéo
prioritaire et l'état de santé auto-déclaré des participants. La pratique du
jeu vidéo prioritaire est définie par la fréquence à laquelle les individus
jouent aux jeux vidéo, allant de "jamais" à "très souvent".

\begin{figure}[H]
    \centering
    \includegraphics[width=0.8\textwidth]{Images/JV_prioritaire_ADRS.pdf}
    \caption{Répartition des individus selon la pratique du jeu vidéo prioritaire par score ADRS.}
    \label{fig:jv_adrs}
\end{figure}

Le graphique \ref{fig:jv_adrs} montre la répartition des individus selon leur
score ADRS et leur fréquence de jeu vidéo. On observe que les participants
ayant un risque d'EDC peu élevé (74.3\%) ou modéré (68.9\%) sont
majoritairement des joueurs qui ne jouent "jamais" ou "rarement". En revanche,
ceux ayant un risque d'EDC important (61.3\%) montrent une proportion plus
élevée de joueurs fréquents ("assez souvent" ou "très souvent").

\begin{figure}[H]
    \centering
    \includegraphics[width=0.8\textwidth]{Images/JV_prioritaire_etat_sante.pdf}
    \caption{Répartition des individus selon la pratique du jeu vidéo prioritaire par état de santé.}
    \label{fig:jv_sante}
\end{figure}

Le graphique \ref{fig:jv_sante} illustre la relation entre la fréquence de jeu
vidéo et l'état de santé auto-déclaré. Les participants déclarant un état de
santé "très satisfaisant" (75\%) ou "plutôt satisfaisant" (69.6\%) sont
principalement des joueurs qui ne jouent "jamais" ou "rarement". À l'inverse,
ceux déclarant un état de santé "peu satisfaisant" (65\%) ou "pas du tout
satisfaisant" (66.7\%) incluent une proportion plus élevée de joueurs
fréquents.

Ces résultats suggèrent une corrélation entre une pratique fréquente du jeu
vidéo et une perception moins positive de l'état de santé. Cependant, il est
important de noter que cette relation est corrélationnelle et ne prouve pas la
causalité. D'autres facteurs pourraient influencer à la fois la fréquence de
jeu et la perception de la santé.


\subsection{Les jeux d'argent et leurs effets}

Les jeux d'argent peuvent causer des problèmes de santé, des problèmes
financiers, des critiques de la part de l'entourage ou même un sentiment de
culpabilité. Ces conséquences sont liées à la santé mentale de l'individu. En
effet, d'après la Recherche en santé mentale Canada (RSMC), les problèmes
économiques affectent négativement la santé mentale. La culpabilité est
elle-même un facteur de dépression, d'après la Haute Autorité de Santé. Il
semble donc pertinent de réaliser une étude du lien entre toutes ces
conséquences de la pratique de jeux d'argent. 


Pour cela, nous étudions l'ACM ci-dessous. Les variables suivantes ont été
sélectionnées:
\begin{itemize}
    \item Le jeu a causé des problèmes de santé (jeuPbSante)
    \item Les habitudes de jeu ont été critiquées par d'autres personnes
    (critique)
    \item Le jeu a causé des difficultés financières (pbArgent)
    \item \item Les habitudes de jeu ont causées un sentiment de culpabilité
    (culpabilite)
    \item La catégorie ADRS (ADRS\_cat)
\end{itemize}
Pour les trois premières variables, les modalités vont de 1 à 4 et signifient:
\begin{itemize}
    \item 1: "Jamais"
    \item 2: "Parfois"
    \item 3: "La plupart du temps"
    \item 4: "Presque toujours"
\end{itemize}
Pour le score ADRS, nous rappelons:
\begin{itemize}
    \item 0: "Sans risque de dépression"
    \item 1: "Risque modéré de dépression"
    \item 2: "Risque élevé de dépression"
\end{itemize}

\begin{figure}[H]
    \centering
    \includegraphics[width=1\textwidth]{Images/P3/ACM1_dim12.png}
    \caption{Interactions entre les conséquences des jeux d'argent sur la santé mentale}
    \label{fig: ACMP3}
\end{figure}

Les axes de cette ACM ne résument qu'une partie limitée de l'information
(31,9\% au total). Nous aurions pu prendre en considération la dimension 3,
mais celle-ci n'apporte pas à notre analyse.

\vskip 10pt

L'axe 1 est principalement structuré par les modalités 4 de l'ensemble des
variables liées aux conséquences de la pratique des jeux d'argent ("Presque
toujours"). La proximité de ces modalités indiquent une forte co-occurrence des
effets négatifs du jeu. En revanche, les modalités de la variable ADRS sont
toutes situées à proximité de l'origine. On ne peut donc pas conclure, à partir
de ce graphique, qu'il existe une association entre les effets négatifs du jeu
et le niveau de risque dépressif.

\vskip 10pt

L'axe vertical permet quant à lui de distinguer les modalités intermédiaires
des varialbles. On observe notamment que les modalités 3 ("La plupart du
temps") sont plus hautes sur cet axe, suivis par les modalités 2 ("Parfois").
Cette organisation confirme l'existence d'un gradient dans la fréquence des
effets du jeu: les réponses intermédiaires sont cohérentes entre elles et
traduisent des conséquences modérées, distinct des réponses extrêmes qui sont
l'absence ou l'intensité maximale des conséquences étudiées.

\subsection{Discussion des résultats}

Cette analyse met en évidence plusieurs corrélations entre la pratique des
jeux, qu'ils soient vidéo ou d'argent, et la santé mentale auto-déclarée des
participants. Les résultats suggèrent qu'une pratique plus fréquente des jeux,
notamment des jeux d'argent, est associée à une perception plus négative de la
santé mentale, ce qui pourrait indiquer un impact défavorable sur le bien-être
des individus. Bien que la relation entre la pratique des jeux vidéo et la
santé mentale soit également présente, elle semble moins marquée.

L'analyse des scores ADRS révèle qu'une proportion significative des
participants présente un faible risque de dépression, mais une minorité
d'individus se trouve dans des catégories de risque plus élevé. Ces résultats
soulignent l'importance de prendre en compte la diversité des expériences des
joueurs et de considérer des facteurs supplémentaires qui pourraient influencer
la santé mentale, comme les habitudes de vie, les conditions sociales et
économiques, ou encore la manière dont les jeux sont vécus par chaque individu.

Il est essentiel de noter que, bien que des corrélations aient été observées,
cela ne signifie pas nécessairement qu'une causalité existe entre la fréquence
des jeux et la dégradation de l'état de santé mentale. Des recherches
supplémentaires seraient nécessaires pour mieux comprendre les mécanismes
sous-jacents et les facteurs contribuant à cette relation.
