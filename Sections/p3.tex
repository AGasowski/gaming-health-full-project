\section{Liens entre la santé mentale et la consommation de jeux}

"Un esprit sain dans un corps sain". Après s'être intéressé au lien entre la
consommation de jeux et la santé physique, intéressons-nous maintenant à la
santé mentale.


Un indicateur se prête particulièrement à l'étude de la santé mentale:
l'Adolescent Depression Rating Scale (ADRS). Cette échelle de mesure allant de
0 à 10 permet de calculer le risque d'Épisode Dépressif Caractérisé (EDC).
Cette échelle est représentée dans le graphique \ref{fig:adrs}, illustrant la
distribution des risques de dépression parmi la population sondée.

\begin{figure}[H]
    \centering
    \includegraphics[width=0.8\textwidth]{Images/P3/scoreADRSdistrib.png}
    \caption{Proportions de chaque score ADRS.}
    \label{fig:adrs}
    \textit{Note de lecture: les individus ayant obtenu un score ADRS de 1
    représentent 17\% de la population.}\\
    \textit{Source: Enquête ESCAPAD 2022.}
\end{figure}


Ainsi une proportion significative des participants (68\%) présente un risque
de dépression faible (score ADRS < 4), tandis qu'une part non négligeable (10\%
des jeunes) est soumise à un risque d'EDC important (score ADRS \(\geq 7\))
(Figure \ref{fig:adrs}).

\subsection{Contrôle du jeu vidéo et santé mentale}
Plus de la moitié (55,4\%) des individus à risques jouent aux jeux vidéo toutes
les semaines voire tous les jours (Figure \ref{jv_EDC}). On pourrait penser que
les joueurs réguliers sont donc surreprésentés. Pourtant la part de la
population entière à jouer à une telle fréquence est de 54,9\% (Figure
\ref{freq jv}). Cette infime différence ne laisse donc pas penser qu'il y a un
lien notable entre la consommation de jeu vidéo et un risque élevé d'Episode
Dépressif Caractérisé.

\begin{figure}[H]
    \centering
    \includegraphics[width=0.8\textwidth]{Images/P3/freq_EDC_eleve.png}
    \caption{Répartition des individus à risque élevé d'EDC selon la fréquence
    de consommation de jeux vidéo.}
    \label{jv_EDC}
    \textit{Note de lecture: 33,4\% des individus présentant des risques élevés
    d'EDC jouent aux jeux vidéo toutes les semaines}\\
    \textit{Source: Enquête ESCAPAD 2022.}
\end{figure}

Mais l'addiction n'est pas seulement caractérisée par la fréquence de
consommation. Elle est aussi définie par le contrôle que l'on a sur sa
pratique. Ne plus être capable de discerner ses priorités est à la fois signe
d'addiction et de mal être.  
Nous entendrons par pratique du jeu prioritaire l'occurrence à laquelle les
individus ont "accordé une priorité plus importante aux jeux vidéo par rapport
aux autres intérêts de la vie et aux activités quotidiennes" (ESCAPAD, 2022).
Ces occurrences sont catégorisées de "jamais" à "souvent". Il existe bien un
lien statistique entre cette variable et la catégorie ADRS, d'après le test du
khi\textsuperscript{2} en \hyperref[ann:correlation]{Annexe C}. Même si le V de
Cramer nous indique une très faible intensité de l'association, nous arrivons
à observer certaines tendances.


\begin{figure}[H]
    \centering
    \includegraphics[width=0.8\textwidth]{Images/P3/JV_priorite.png}
    \caption{Répartition des individus selon la pratique du jeu vidéo prioritaire selon le risque d'EDC}
    \label{jv_priorite}
    \textit{Note de lecture: 16,2\% des individus présentant un risque d'EDC
    élevé accordent une priorité plus importante aux jeux vidéo qu'au reste}\\
    \textit{Source: Enquête ESCAPAD 2022.}
\end{figure}


\vskip10pt

\begin{figure}[H]
    \centering
    \includegraphics[width=0.8\textwidth]{Images/P3/JV_controle_EDC.png}
    \caption{Risque d'EDC selon la fréquence de difficulté à contrôler sa
    consommation de jeux vidéo.}
    \label{jv_controle}
    \textit{Note de lecture: 65,9\% des individus ayant parfois des difficultés
    à contrôler leur consommation de jeux vidéo ne présentent aucun risque
    d'EDC}\\
    \textit{Source: Enquête ESCAPAD 2022.}
\end{figure}

La part d'individus dont la pratique du jeu vidéo a "parfois" pris la priorité
sur les autres aspects de la vie est constante peu importe le risque d'Episode
Dépressif Caractérisé (Figure \ref{jv_priorite}). En revanche plus le risque
d'EDC est élevé, plus la part d'individus ayant souvent priorisé le jeu vidéo
sur le reste est importante. Ces derniers ne représentent que 4,4\% des
individus sans risque d'EDC, tandis que cette part est presque 4 fois plus
importante pour ceux présentant un risque élevé: ils sont 16,2\% (Figure
\ref{jv_priorite}).


\vskip10pt

Plus il est récurrent pour un individu d'avoir des difficultés à contrôler sa
consommation de jeux vidéo, plus celui-ci est exposé à des risques élevés
d'EDC. En effet, seulement 45\% de ceux qui ont souvent des difficultés à
contrôler leur pratique ne présentent aucun risque d'EDC alors qu'ils sont 70\%
parmi ceux qui ne perdent jamais le contrôle (Figure \ref{jv_controle}). Ce
chiffre est de 68\% dans l'ensemble de la population (Figure \ref{fig:adrs}) A
noter que les différences  de risque d'EDC sont plus importantes entre ceux qui
ont souvent des difficultés de contrôle et ceux qui en ont parfois, qu'entre
ceux qui n'en ont jamais et ceux qui en ont parfois. Les colonnes "jamais" et
"parfois" sont presque similaires (Figure \ref{jv_controle}), quand bien même
les non-joueurs sont logiquement compris dans la colonne "jamais". \\
On aurait pu penser que du fait de la présence de cette population, les risques
d'EDC seraient répartis différemment selon que les individus ont parfois ou
jamais du mal à contrôler leur consommation. Cela laisse penser que le lien
entre la consommation contrôlée de jeux vidéo et le risque d'EDC est minime. En
revanche il y a un lien clair entre la possibilité de gérer raisonnablement sa
consommation de jeux vidéo et le risque d'Episode Dépressif Caractérisé. 




\subsection{Jeux d'argent et effets perçus}

Les jeux d'argent peuvent causer des problèmes de santé, des problèmes
financiers, des critiques de la part de l'entourage ou même un sentiment de
culpabilité. Ces conséquences sont liées à la santé mentale de l'individu. En
effet, d'après la Recherche en santé mentale Canada (Le Vecteur, 2025
\cite{CCSM2025}), les problèmes économiques affectent négativement la santé
mentale. La culpabilité est elle-même un facteur de dépression, d'après la
Haute Autorité de Santé (2017 \cite{HAS2017}). Il semble donc pertinent de
réaliser une étude du lien entre toutes ces conséquences de la pratique de jeux
d'argent. Ici notre but n'est pas d'établir des relations de causalité. Nous
constatons simplement des liens, s'ils existent, entre différents éléments.


Pour cela, nous étudions l'ACM ci-dessous. Les variables suivantes ont été
sélectionnées:
\begin{itemize}
    \item L'individu considère que le jeu a causé des problèmes de santé
    (jeuPbSante)
    \item Les habitudes de jeu ont été critiquées par d'autres personnes
    (critique)
    \item Le jeu a causé des difficultés financières (pbArgent)
    \item Les habitudes de jeu ont causées un sentiment de culpabilité
    (culpabilite)
    \item La catégorie ADRS (ADRS\_cat)
\end{itemize}
Pour les trois premières variables, les modalités vont de 1 à 4 et signifient:
\begin{itemize}
    \item 1: "Jamais"
    \item 2: "Parfois"
    \item 3: "La plupart du temps"
    \item 4: "Presque toujours"
\end{itemize}
Pour le score ADRS, nous rappelons:
\begin{itemize}
    \item 0: "Sans risque de dépression"
    \item 1: "Risque modéré de dépression"
    \item 2: "Risque élevé de dépression"
\end{itemize}

\begin{figure}[H]
    \centering
    \includegraphics[width=1\textwidth]{Images/P3/ACM1_dim12.png}
    \caption{Interactions entre les conséquences des jeux d'argent et la santé mentale}
    \label{fig: ACMP3}
    \textit{Source: Enquête ESCAPAD 2022.}
\end{figure}

Les axes de cette ACM ne résument qu'une partie limitée de l'information
(31,9\% au total). Nous aurions pu prendre en considération la dimension 3,
d'après le critère du coude (\hyperref[ann:coude2]{Annexe F}), mais celle-ci
n'améliore pas notre analyse (voir \hyperref[ann:dim13]{Annexe G}).

\vskip 10pt

L'axe 1 est principalement structuré par les modalités 4 de l'ensemble des
variables liées à la pratique des jeux d'argent ("Presque toujours"). La
proximité de ces modalités indiquent une forte co-occurrence des présumés
effets négatifs des jeux d'argent. En revanche, les modalités de la variable
ADRS sont toutes situées à proximité de l'origine. On ne peut donc pas
conclure, à partir de ce graphique, qu'il existe une association entre les
présumés effets négatifs du jeu et le niveau de risque dépressif.

\vskip 10pt

L'axe vertical permet quant à lui de distinguer les modalités intermédiaires
des variables. On observe notamment que les modalités 3 ("La plupart du temps")
sont plus hautes sur cet axe, suivies par les modalités 2 ("Parfois"). Cette
organisation confirme l'existence d'un gradient dans la fréquence des effets du
jeu: les réponses intermédiaires sont cohérentes entre elles et traduisent des
conséquences modérées, distinctes des réponses extrêmes qui sont l'absence ou
l'intensité maximale des conséquences présumées étudiées.

