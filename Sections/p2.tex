\section{Deuxième partie}

\subsection{Première sous-partie}

L'OMS définit la santé physique comme étant "un état de complet bien-être physique". Il s'agit d'un
niveau de bien-être permettant au corps de réaliser des activités physiques dans les meilleures
conditions. On peut dès lors essayer de qualifier le niveau de santé physique d'un individu, en 
nous intéressant à des marqueurs de cette santé.\\

L'IMC est un premier marqueur de la santé physique, faisant le lien entre la taille et le poids
pour donner un aperçu de la morphologie. L'OMS classe les individus en 4 grandes catégories d'IMC:
insuffisance pondérale, corpulence normale, surpoids, obésité.\\

\vskip 10pt

\begin{center}\includegraphics[width=0.7\textwidth]{Images/Freq_JA_IMC.png}\\[2.0 cm] 

Pour cette analyse, tous les types de jeux d'argent ont été regroupés. On remarque une variation la
fréquence de jeux d'argent des individus en fonction de leur catégorie d'IMC. En effet, les
individus en obésité ont une plus grande tendance à jouer aux jeux d'argent, avec
30,5\% de joueurs parmis eux. A l'opposé, les individus en insuffisance pondéral sont ceux qui jouent
le moins avec 27,3\% de joueurs. Entre les deux se trouvent les individus à la corpulance normale et
en surpoids, avec respectivement 27,9\% et 28,4\% de joueurs.\\

La même étude a été faite en ce qui concerne les jeux vidéos.

\begin{center}\includegraphics[width=0.7\textwidth]{Images/Freq_JV_IMC.png}\\[2.0 cm]

Le résultat semble plus apparent ici puisque le nombre de joueurs de jeux vidéos est plus important.
Les individus en obésité comptent parmis eux plus d'un quart de joueurs quotidiens (26,6\%), plus
que toute autre catégorie d'IMC. Ce sont ensuite les individus en insuffisance corporelle qui jouent
le plus au quotidien (20,6\% d'entre eux). Ce résultat est différent que pour les jeux d'argent,
puisque les individus en insuffisance pondérale n'était pas plus grands joueurs de jeux d'argent.
Ils se démarquent en ce qui concerne les jeux vidéos.\\




\newpage


\subsection{Deuxième sous-partie}

Le comportement des joueurs est un élément d'étude important, car au-delà d'un simple constat sur
un individu, il donne des habitudes de sa vie que l'on peut mettre en relation avec ses habitudes
de jeu.\\

La pratique sportive est 

\vskip 10pt

\begin{figure}[htb]
    \captionsetup{justification=centering}
    \begin{centering}
        \centerline{\includegraphics[scale = 0.6]{Images/image4.pdf}}
        \caption{Titre. \\
        \textit{Note de lecture: xxxxxxxxxxxxx.} \\
        \textit{Champ et source: xxxxxxxxxxxxx.}}
    \end{centering}
\end{figure}

\newpage

\blindtext[2]

\newpage

\subsection{Troisième sous-partie}