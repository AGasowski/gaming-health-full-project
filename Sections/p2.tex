\section{Deuxième partie}

\subsection{Première sous-partie}

L'OMS définit la santé physique comme étant "un état de complet bien-être physique". Il s'agit d'un
niveau de bien-être permettant au corps de réaliser des activités physiques dans les meilleures
conditions. On peut dès lors essayer de qualifier le niveau de santé physique d'un individu, en 
nous intéressant à des marqueurs de cette santé.\\

Chaque individu a répondu au questionnaire en qualifiant son état de santé. 46,4\% des individus
qualifient leur état de santé comme très satisfaisant et 8,6\% comme peu ou pas du tout satisfaisant.
Chez les joueurs quotidiens de jeux vidéos, on voit une différence de l'état de santé général,
puisque 41,8\% déclarent avoir un état de santé très satisfaisant et 13,1\% comme peu ou pas du tout
satisfaisant.\\

En ce qui concerne les joueurs quotidiens de jeux d'argent, 47,1\% qualifient leur état de santé
de très satisfaisant, ce qui est semblable au reste de la population. Cependant, les jeux d'argent
ont un réel impact négatif sur l'état de santé des plus gros joueurs puisque 29,4\% d'entre eux
qualifient leur état de santé de peu ou pas du tout satisfaisant. La différence avec le reste de 
la population est très significatif.\\

\begin{table}[ht]
    \centering
    \captionsetup{justification=centering}
    \begin{tabular}{lccc}
          \hline
          Etat de santé & Population générale & \makecell{Joueurs quotidiens\\de jeux vidéo} & \makecell{Joueurs quotidiens\\de jeux d'argent}\\ 
          \hline
          Très satisfaisant & 46,4\% & 41,8\% & 47,1\% \\ 
          \hline
          \makecell{Peu ou pas du\\tout satisfaisant} & 8,6\% & 13,1\% & 29,4\% \\ 
           \hline
    \end{tabular}
    \vskip 10pt
    \caption{Titre.\\
    \textit{Note de lecture: xxxxxxxx.}\\
    \textit{Champ et source: xxxxxxxx.}}
\end{table}

L'IMC est un marqueur de la santé physique, faisant le lien entre la taille et le poids
pour donner un aperçu de la morphologie. L'OMS classe les individus en 4 grandes catégories d'IMC:
insuffisance pondérale, corpulence normale, surpoids, obésité.\\

Pour cette analyse, tous les types de jeux d'argent ont été regroupés. On a retiré les individus qui
ne jouent jamais pour une meilleure lisibilité du graphique. Ces individus ont néanmoins été
pris en compte dans le calcul des pourcentage. 

\begin{center}\includegraphics[width=1\textwidth]{Images/Freq_JA_IMC.png}\\[2.0 cm]\end{center}

On remarque une variation de la fréquence de jeux d'argent des individus en fonction de leur
catégorie d'IMC. En effet, les individus en obésité ont une plus grande tendance à jouer aux jeux 
d'argent, avec 30,5\% de joueurs parmis eux. A l'opposé, les individus en insuffisance pondéral 
sont ceux qui jouent le moins avec 27,3\% de joueurs. Entre les deux se trouvent les individus à 
la corpulance normale et en surpoids, avec respectivement 27,9\% et 28,4\% de joueurs.\\

La même étude a été faite en ce qui concerne les jeux vidéos.

\begin{center}\includegraphics[width=1\textwidth]{Images/Freq_JV_IMC.png}\\[2.0 cm]\end{center}

Les individus en obésité comptent parmis eux plus d'un quart de joueurs quotidiens (26,6\%), plus
que toute autre catégorie d'IMC. Ce sont ensuite les individus en insuffisance corporelle qui jouent
le plus au quotidien (20,6\% d'entre eux). Ce résultat est différent de celui qui concerne les jeux
d'argent, puisque les individus en insuffisance pondérale n'était pas plus grands joueurs de jeux 
d'argent. Ils se démarquent en ce qui concerne les jeux vidéos.\\




\newpage


\subsection{Comportement des joueurs vis à vis de leur santé}

Le comportement des joueurs vis à vis de leur santé est un élément d'étude important, car au-delà 
d'un simple constat sur l'état de santé d'un individu, il donne des habitudes de sa vie que l'on
peut mettre en relation avec ses habitudes de jeu.\\

\begin{center}\includegraphics[width=1\textwidth]{Images/Freq_JA_sport.png}\\[2.0 cm]\end{center}

Plus un individu fait du sport régulièrement, plus il joue aux jeux d'argent. En effet, 34,9\% des
sportifs quotidiens jouent à des jeux d'argent, contre 23,2\% de joueurs chez ceux qui ne pratiquent
pas de sport. \\

Ce résultat est à mettre en parallèle avec celui concernant seuelement les joueurs de paris sportifs,
qui sont des jeux dont l'intérêt est lié au sport.

\begin{center}\includegraphics[width=1\textwidth]{Images/Freq_PS_sport.png}\\[2.0 cm]\end{center}

La différence entre un sportif régulier et un individu moins sportif est plus prononcé sur ce 
graphique. 21,1\% des sportifs quotidiens jouent aux paris sportifs, alors que seulement 5,6\%
des non sportifs parient sur le sport. De plus les sportifs quotidiens sont des joueurs réguliers
de paris sportifs, car 6,8\% d'entre eux parient au moins une fois par semaine.\\

En ce qui concerne les liens entre la pratique de jeux vidéos et la pratique sportive, ceux-ci sont
moins évidents à établir.

\begin{center}\includegraphics[width=1\textwidth]{Images/Freq_JV_sport.png}\\[2.0 cm]\end{center}

Les joueurs quotidiens sont de nouveau ceux qui jouent le plus aux jeux vidéos (85,6\% d'entre eux
jouent). A l'opposé, les non sportifs comptent moins de joueurs parmis eux (76,5\%), mais une grande
partie d'entre eux sont des joueurs très réguliers: 20,3\% des non sportifs jouent presque tous les
jours aux jeux vidéo. C'est la plus grande part de joueurs très réguliers parmi les quatre catégories
de sportifs.\\

Il est également intéressant de comprendre le comportement des individus en ce qui concerne la prise
de rendez-vous chez le médecin. 86,8\% des individus ayant répondu au questionnaire sont allé chez le
médecin dans l'année. Ce chiffre varie si on prend étudie chaque catégorie de joueurs.

\begin{center}\includegraphics[width=1\textwidth]{Images/Freq_JA_medecin.png}\\[2.0 cm]\end{center}

Les plus gros joueurs de jeux d'argent sont allé moins souvent chez le médecin que le reste de la
population. 79,4\% y sont allé, ce qui est au moins 7 points de moins que tout autre catégorie.

\begin{center}\includegraphics[width=1\textwidth]{Images/Freq_JV_medecin.png}\\[2.0 cm]\end{center}

Le même constat, mais moins prononcé, est visible en ce qui concerne les gros joueurs de jeux vidéo.
Cette fois-ci, on voit une légère augmentation de la part de visites chez le médecin avec la
diminution de la fréquence de jeux vidéos. Les joueurs quotidiens se rapprochent cette fois du
comportement du reste de la population puisque 84,6\% d'entre eux se sont rendus chez le médecin.



\vskip 10pt

\newpage


\newpage

\subsection{Troisième sous-partie}