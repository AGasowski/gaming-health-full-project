\section{Liens entre la santé physique et la consommation de jeux}

\subsection{Examen médical des joueurs}

L'OMS définit la santé physique comme étant "un état de complet bien-être
physique". Il s'agit d'un niveau de bien-être permettant au corps de réaliser
des activités physiques dans les meilleures conditions. On peut dès lors
essayer de qualifier le niveau de santé physique d'un individu, en nous
intéressant à des marqueurs de cette santé.

\vspace{1em}

Chaque individu a répondu au questionnaire en évaluant son état de santé. En
réalisant un test du khi\textsuperscript{2} entre la variable qualifiant l'état
de santé et la fréquence de jeu (voir \hyperref[ann:correlation]{Annexe C}), on
remarque qu'elles ont un lien. Un V de Celui-ci est observable dans le tableau
ci-dessous.

\begin{table}[ht]
    \centering
    \captionsetup{justification=centering}
    \begin{tabular}{lccc}
          \hline
          Etat de santé & Population générale & \makecell{Joueurs
          quotidiens\\de jeux vidéo} & \makecell{Joueurs quotidiens\\de jeux
          d'argent}\\ 
          \hline
          Très ou plutôt\\satisfaisant & 91,3\% & 86,9\% & 70,6\% \\
          \hline
          \makecell{Peu ou pas du\\tout satisfaisant} & 8,7\% & 13,1\% & 29,4\%
          \\ 
    
    \end{tabular}
    \vskip 10pt \caption{Répartition de l'état de santé\\
    \textit{Source: Enquête ESCAPAD 2022.}}
    \label{fig:satis_sante}
\end{table}

\vspace{1em}

Parmi l'ensemble des individus interrogés, 8,7\% ne sont pas satisfaits par
leur état de santé. Chez les joueurs quotidiens de jeux vidéo, on voit une
différence de l'état de santé général, puisque 13,1\% déclarent leur état de
santé comme n'étant pas satisfaisant. Les individus jouant aux jeux vidéo
semblent donc avoir un état de santé moins satisfaisant que la population
générale.

\vspace{1em}

De leur côté, les joueurs quotidiens de jeux d'argent sont encore moins
satisfaits de leur état de santé. En effet, 29,4\% de cette catégorie de
joueurs déclare ne pas être satisfait de son état de santé. La différence de
21,7 points de pourcentage avec la population générale est très significative.

\vspace{1em}

L'IMC est également un marqueur de la santé physique, faisant le lien entre la
taille et le poids pour donner un aperçu de la morphologie. L'OMS classe les
individus en 4 grandes catégories d'IMC: insuffisance pondérale, corpulence
normale, surpoids, obésité.

Pour cette analyse, tous les types de jeux d'argent ont été regroupés. On a
seulement pris en compte les individus ayant déjà joué dans cette étude.

\begin{figure}[H]
    \centering
    \includegraphics[width=1\textwidth]{Images/P2/JA_IMC.png}
    \caption{Proportion de joueurs de jeux d'argent et IMC.}
    \label{fig:JA_IMC}
    \textit{Source: Enquête ESCAPAD 2022.}
\end{figure}

Les test d'indépendance du chi2 nous indique qu'il n'y a pas de lien entre
l'IMC et la fréquence de jeux d'argent (voir \hyperref[ann:correlation]{Annexe
C}). On remarque en effet une augmentation négligeable de la pratique de jeux
d'argent des individus en obésité. Ces derniers comptent 30,5\% de joueurs
parmi eux. A l'opposé, les individus en insuffisance pondéral sont ceux qui
jouent le moins avec 27,3\% de joueurs. Entre les deux se trouvent les
individus à la corpulance normale et en surpoids, avec respectivement 27,9\% et
28,4\% de joueurs (Tableau \ref{fig:JA_IMC}). L'écart entre ces pourcentage
n'est pas significatif et ne nous indique aucun lien entre l'IMC d'un individu
et sa pratique de jeux d'argent. En ce qui concerne la fréquence de jeu, on
observe qu'elle augmente très légèrement avec l'IMC.

\begin{figure}[H]
    \centering
    \includegraphics[width=1\textwidth]{Images/P2/JA_freqsimp_joueurs_IMC.png}
    \caption{Fréquence de jeux d'argent et IMC.}
    \label{fig:freq_JA_IMC}
    \textit{Source: Enquête ESCAPAD 2022.}
\end{figure}

Les pourcentages de joueurs quotidiens sont trop faibles pour pouvoir en tirer
une réelle observation. En revanche, En ce qui concerne les joueurs
hebdomadaires et quotidiens, on peut dire qu'ils sont plus présents parmi les
individus en surpoids (19,6\% au total) et les individus en obésité (17,6\% au
total), que parmi les individus en insuffisance pondérale (9,6\% au total) ou
de corpulence normale (14,9\% au total) (Figure \ref{fig:freq_JA_IMC}).

\vspace{1em}

Une étude similaire a été réalisée en ce qui concerne la pratique de jeux
vidéo, et son lien avec l'IMC. De nouveau, le test du chi2 nous indique une
indépendance entre les deux variables (voir \hyperref[ann:correlation]{Annexe
C}).

\begin{figure}[H]
    \centering
    \includegraphics[width=1\textwidth]{Images/P2/JV_IMC.png}
    \caption{Proportion de joueurs de jeux vidéo en fonction de l'IMC.}
    \label{fig:JV_IMC}
    \textit{Source: Enquête ESCAPAD 2022.}
\end{figure}

On remarque que les plus grands joueurs de jeux vidéo sont les personnes en
obésité, qui jouent pour 84,7\% d'entre elles. La différence avec les autres
catégories d'IMC, qui se situent entre 80,2\% et 81,7\% est plutôt faible
(Figure \ref{fig:JV_IMC}). On ne remarque pas de corrélation intéressante.
L'IMC et la pratique de jeux vidéos ne semblent donc pas liés.

\vspace{1em}

En revanche, la différence est plus visible lorsqu'il s'agit de la fréquence de
jeu. D'après le test du chi2, il existe un lien entre la fréquence de la
pratique de jeux vidéo et l'IMC (voir \hyperref[ann:correlation]{Annexe C}).

\begin{figure}[H]
    \centering
    \includegraphics[width=1\textwidth]{Images/P2/JV_freqsimp_joueurs_IMC.png}
    \caption{Fréquence de jeu et IMC.}
    \label{fig:freq_JV_IMC}
    \textit{Source: Enquête ESCAPAD 2022.}
\end{figure}

Presqu'un tier (31,4\%) des joueurs obèses jouent tous les jours , c'est plus
que toute autre catégorie d'IMC. Ce sont ensuite les joueurs en insuffisance
pondérale qui jouent le plus au quotidien (25,2\% d'entre eux). Ce résultat est
différent de celui qui concerne les jeux d'argent, puisque les individus en
insuffisance pondérale ne sont pas des joueurs réguliers de jeux d'argent. Ils
se démarquent en ce qui concerne les jeux vidéo. On en conclut que les joueurs
les plus réguliers de jeux vidéo sont les individus ayant des IMC aux
extrémités, c'est-à-dire les joueurs en insuffisance pondérale et en obésité.
Il est également intéressant de noter que ce sont les individus de corpulence
normale qui jouent le moins au quotidien.




\newpage


\subsection{Comportement des joueurs vis à vis de leur santé}

Le comportement des joueurs vis à vis de leur santé est un élément d'étude
important, car au-delà d'un simple constat sur l'état de santé d'un individu,
il donne des habitudes de vie que l'on peut mettre en relation avec des
habitudes de jeu. D'après le test du chi2, il y a bien un lien de corrélation
entre la pratique sportive et la consommations de jeux (voir
\hyperref[ann:correlation]{Annexe C}).

\begin{figure}[H]
    \centering
    \includegraphics[width=1\textwidth]{Images/P2/JA&PS&sans_sport.png}
    \caption{Répartition des sportifs parmi les joueurs.}
    \label{fig:freq_joueur_JA_sport}
    \textit{Source: Enquête ESCAPAD 2022.}
\end{figure}

Un individu jouant aux jeux d'argent est plus susceptible de faire du sport
(71,3\%) qu'un individu de jouant pas (63,9\%) (Figure
\ref{fig:freq_joueur_JA_sport}). On remarque ainsi que ces deux activités sont
liées.

\vspace{1em}

En poussant la recherche, on se rend compte qu'en réalité, ce sont les paris
sportifs qui influencent largement ce résultat. En effet, 83,4\% des individus
jouant aux paris sportifs font du sport (Figure
\ref{fig:freq_joueur_JA_sport}).

Ce résultat semble montrer qu'au-delà du lien entre la pratique des jeux
d'argent et la pratique sportive, il y a un lien bien plus fort entre la
pratique de paris sportifs et la pratique sportive. Ce résultat semble nous
dire que les deux activités sont liées, ce qui semble intuitif puisqu'elles
s'intéressent toutes les deux au sport.

\vspace{1em}

En ce qui concerne les liens entre la pratique de jeux vidéo et la pratique
sportive, ceux-ci sont moins évidents à établir.

\begin{figure}[H]
    \centering
    \includegraphics[width=1\textwidth]{Images/P2/JV_sport.png}
    \caption{Répartition des sportifs parmi les joueurs.}
    \label{fig:prop_JV_sport}
    \textit{Source: Enquête ESCAPAD 2022.}
\end{figure}

Les sportifs quotidiens sont de nouveau ceux qui jouent le plus aux jeux vidéo
(85,6\% d'entre eux jouent). A l'opposé, les non sportifs comptent moins de
joueurs parmis eux (76,5\%) (Figure \ref{fig:prop_JV_sport}), mais une grande
partie d'entre eux sont des joueurs très réguliers: 20,3\% des non sportifs
jouent quotidiennement aux jeux vidéo (Figure \ref{fig:freq_joueur_JV_sport}).
C'est la plus grande part de joueurs quotidiens parmi les quatre catégories de
sportifs.

\begin{figure}[H]
    \centering
    \includegraphics[width=1\textwidth]{Images/P2/JVfreqsimp_sport.png}
    \caption{Répartition des sportifs parmi les joueurs.}
    \label{fig:freq_joueur_JV_sport}
    \textit{Source: Enquête ESCAPAD 2022.}
\end{figure}


Il est également intéressant de comprendre le comportement des individus en ce
qui concerne la prise de rendez-vous chez le médecin, qui est un indicateur
d'un individu qui prend soin de sa santé ou non. 86,8\% des individus ayant
répondu au questionnaire sont allé chez le médecin dans l'année (Enquête
ESCAPAD 2022). Ce chiffre varie si on étudie chaque catégorie de joueurs.

\begin{figure}[H]
    \centering
    \includegraphics[width=1\textwidth]{Images/P2/JA_medecin.png}
    \caption{Part des individus ayant été chez le médecin au cours de l'année pour chaque catégorie de joueurs de jeux d'argent.}
    \label{fig:freq_medJV}
    \textit{Source: Enquête ESCAPAD 2022.}
\end{figure}

Les plus gros joueurs de jeux d'argent sont allé moins souvent chez le médecin
que le reste de la population. 79,4\% y sont allé, ce qui est au moins 7 points
de moins que tout autre catégorie (Figure \ref{fig:freq_medJV}). Si on
considère que les quatre catégories ont des états de santé similaires en
moyenne, alors ils devraient consulter de la même manière un professionnel de
santé. On pourrait déduire de cette observation que les joueurs quotidiens de
jeux d'argent prennent moins soin de leur santé.


\begin{figure}[H]
    \centering
    \includegraphics[width=1\textwidth]{Images/P2/JV_medecin.png}
    \caption{Part des individus ayant été chez le médecin au cours de l'année pour chaque catégorie de joueurs de jeux vidéo.}
    \label{fig:freq_medJA}
    \textit{Source: Enquête ESCAPAD 2022.}
\end{figure}

Le même constat, mais moins prononcé, est visible en ce qui concerne les gros
joueurs de jeux vidéo. Cette fois-ci, on voit une légère augmentation de la
part de visites chez le médecin avec la diminution de la fréquence de jeux
vidéos. Les joueurs quotidiens se rapprochent cette fois du comportement du
reste de la population puisque 84,6\% d'entre eux se sont rendus chez le
médecin, ce qui est 2 points de moins que pour l'ensemble de la population
(86,8\%).