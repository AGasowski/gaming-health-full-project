\section{Deuxième partie}

\subsection{Première sous-partie}

L'OMS définit la santé physique comme étant "un état de complet bien-être physique". Il s'agit d'un
niveau de bien-être permettant au corps de réaliser des activités physiques dans les meilleures
conditions. On peut dès lors essayer de qualifier le niveau de santé physique d'un individu, en 
nous intéressant à des marqueurs de cette santé.\\

L'IMC est un premier marqueur de la santé physique, faisant le lien entre la taille et le poids
pour donner un aperçu de la morphologie. L'OMS classe les individus en 4 grandes catégories d'IMC:
insuffisance pondérale, corpulence normale, surpoids, obésité.\\

\vskip 10pt

\begin{center}\includegraphics[width=1\textwidth]{Images/Freq_JA_IMC.png}\\[2.0 cm] 

Pour cette analyse, tous les types de jeux d'argent ont été regroupés. On a retiré les individus qui
ne jouent jamais pour une meilleure visibilité du graphique. Ces individus ont néanmoins été
pris en compte dans le calcul des pourcentage. On remarque une variation de la
fréquence de jeux d'argent des individus en fonction de leur catégorie d'IMC. En effet, les
individus en obésité ont une plus grande tendance à jouer aux jeux d'argent, avec
30,5\% de joueurs parmis eux. A l'opposé, les individus en insuffisance pondéral sont ceux qui jouent
le moins avec 27,3\% de joueurs. Entre les deux se trouvent les individus à la corpulance normale et
en surpoids, avec respectivement 27,9\% et 28,4\% de joueurs.\\

La même étude a été faite en ce qui concerne les jeux vidéos.

\begin{center}\includegraphics[width=1\textwidth]{Images/Freq_JV_IMC.png}\\[2.0 cm]

Les individus en obésité comptent parmis eux plus d'un quart de joueurs quotidiens (26,6\%), plus
que toute autre catégorie d'IMC. Ce sont ensuite les individus en insuffisance corporelle qui jouent
le plus au quotidien (20,6\% d'entre eux). Ce résultat est différent de celui qui concerne les jeux
d'argent, puisque les individus en insuffisance pondérale n'était pas plus grands joueurs de jeux 
d'argent. Ils se démarquent en ce qui concerne les jeux vidéos.\\




\newpage


\subsection{Comportement des joueurs vis à vis de leur santé}

Le comportement des joueurs vis à vis de leur santé est un élément d'étude important, car au-delà 
d'un simple constat sur l'état de santé d'un individu, il donne des habitudes de sa vie que l'on
peut mettre en relation avec ses habitudes de jeu.\\

\begin{center}\includegraphics[width=1\textwidth]{Images/Freq_JA_sport.png}\\[2.0 cm]

Plus un individu fait du sport régulièrement, plus il joue aux jeux d'argent. En effet, 34,9\% des
sportifs quotidiens jouent à des jeux d'argent, contre 23,2\% de joueurs chez ceux qui ne pratiquent
pas de sport. \\

Ce résultat est à mettre en parallèle avec celui concernant seuelement les joueurs de paris sportifs,
qui sont des jeux dont l'intérêt est lié au sport.

\begin{center}\includegraphics[width=1\textwidth]{Images/Freq_PS_sport.png}\\[2.0 cm]

La différence entre un sportif régulier et un individu moins sportif est plus prononcé sur ce 
graphique. 21,1\% des sportifs quotidiens jouent aux paris sportifs, alors que seulement 5,6\%
des non sportifs parient sur le sport. De plus les sportifs quotidiens sont des joueurs réguliers
de paris sportifs, car 6,8\% d'entre eux parient au moins une fois par semaine.\\


\vskip 10pt

\newpage


\newpage

\subsection{Troisième sous-partie}