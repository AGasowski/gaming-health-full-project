\section{Deuxième partie}

\subsection{Première sous-partie}

L'OMS définit la santé physique comme étant "un état de complet bien-être physique". Il s'agit d'un
niveau de bien-être permettant au corps de réaliser des activités physiques dans les meilleures
conditions. On peut dès lors essayer de qualifier le niveau de santé physique d'un individu, en 
nous intéressant à des marqueurs de cette santé.\\

L'IMC est un premier marqueur de la santé physique, faisant le lien entre la taille et le poids
pour donner un aperçu de la morphologie. L'OMS classe les individus en 4 grandes catégories d'IMC:
insuffisance pondérale, corpulence normale, surpoids, obésité.\\

\vskip 10pt

\begin{table}[ht]
    \centering
    \captionsetup{justification=centering}
    \begin{tabular}{lcc}
          \hline
          Fréquence de jeu & IMC moyen (jeux d'argent) & IMC moyen (jeux vidéos)\\ 
          \hline
          Jamais & 21,73 & 21,78 \\ 
          Une fois par mois ou moins & 21,65 & 21,74 \\ 
          2 à 3 fois par mois & 22,24 & 21,71 \\ 
          1 fois par semaine & 22,39 & 21,59 \\ 
          Plusieurs fois par semaine & 22,59 & 21,75 \\ 
          Tous les jours & 21,61 & 21,93 \\ 
           \hline
    \end{tabular}
    \vskip 10pt
    \caption{IMC moyen en fonction de la fréquence de jeu\\
    \textit{Note de lecture: xxxxxxxx.}\\
    \textit{Champ et source: xxxxxxxx.}}
\end{table}

On remarque une faible variation de l'IMC moyen des joueurs en fonction de leur fréquence de jeu.
L'IMC et la fréquence de jeu ne semblent pas liés dans notre étude.\\



\newpage


\subsection{Deuxième sous-partie}

Le comportement des joueurs est un élément d'étude important, car au-delà d'un simple constat sur
un individu, il donne des habitudes de sa vie que l'on peut mettre en relation avec ses habitudes
de jeu.\\

La pratique sportive est 

\vskip 10pt

\begin{figure}[htb]
    \captionsetup{justification=centering}
    \begin{centering}
        \centerline{\includegraphics[scale = 0.6]{Images/image4.pdf}}
        \caption{Titre. \\
        \textit{Note de lecture: xxxxxxxxxxxxx.} \\
        \textit{Champ et source: xxxxxxxxxxxxx.}}
    \end{centering}
\end{figure}

\newpage

\blindtext[2]

\newpage

\subsection{Troisième sous-partie}