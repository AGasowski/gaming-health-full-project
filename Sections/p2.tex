\section{Liens entre la santé physique et la consommation de jeux}

\subsection{Etat de santé et jeux}

L'OMS définit la santé physique comme étant "un état de complet bien-être
physique". Il s'agit d'un niveau de bien-être permettant au corps de réaliser
des activités physiques dans les meilleures conditions. On peut dès lors
essayer de qualifier le niveau de santé physique d'un individu, en nous
intéressant à des marqueurs de cette santé.

Chaque individu a répondu au questionnaire en qualifiant son état de santé.
Parmi eux, 8,7\% n'en sont pas satisfaits. Chez les joueurs quotidiens de jeux
vidéos, on voit une différence de l'état de santé général, puisque 13,1\%
déclarent leur état de santé comme n'étant pas satisfaisant. Les individus
jouant aux jeux vidéos semblent donc avoir un état de santé moins satisfaisant
que la population générale.

De leur côté, les joueurs quotidiens de jeux d'argent sont encore moins
satisfaits de leur état de santé. En effet, 29,4\% de cette catégorie de
joueurs déclare ne pas être satisfait de son état de santé. La différence avec
le la population générale, de +21,7\%, est très significatif.

\begin{table}[ht]
    \centering
    \captionsetup{justification=centering}
    \begin{tabular}{lccc}
          \hline
          Etat de santé & Population générale & \makecell{Joueurs
          quotidiens\\de jeux vidéo} & \makecell{Joueurs quotidiens\\de jeux
          d'argent}\\ 
          \hline
          Très ou plutôt\\satisfaisant & 91,3\% & 86,9\% & 70,6\% \\
          \hline
          \makecell{Peu ou pas du\\tout satisfaisant} & 8,7\% & 13,1\% & 29,4\%
          \\ 
    
    \end{tabular}
    \vskip 10pt \caption{Répartition de l'état de santé\\
    \textit{Source: Enquête ESCAPAD 2022.}}
    \label{fig:satis_sante}
\end{table}

L'IMC est également un marqueur de la santé physique, faisant le lien entre la
taille et le poids pour donner un aperçu de la morphologie. L'OMS classe les
individus en 4 grandes catégories d'IMC: insuffisance pondérale, corpulence
normale, surpoids, obésité.

Pour cette analyse, tous les types de jeux d'argent ont été regroupés. On a
seulement pris en compte les individus ayant déjà joué dans cette étude.

\begin{figure}[H]
    \centering
    \includegraphics[width=1\textwidth]{Images/P2/JA_IMC.png}
    \caption{Fréquence de jeu et IMC.}
    \label{fig:JA_IMC}
\end{figure}

On remarque une légère augmentation de la pratique de jeux d'argent des
individus avec leur IMC. En effet, les individus en obésité ont une plus grande
tendance à jouer aux jeux d'argent, avec 30,5\% de joueurs parmi eux. A
l'opposé, les individus en insuffisance pondéral sont ceux qui jouent le moins
avec 27,3\% de joueurs. Entre les deux se trouvent les individus à la
corpulance normale et en surpoids, avec respectivement 27,9\% et 28,4\% de
joueurs.

En ce qui concerne la fréquence de jeu, on observe qu'elle augmente également
avec l'IMC.

\begin{figure}[H]
    \centering
    \includegraphics[width=1\textwidth]{Images/P2/JAfreqsimp_IMC.png}
    \caption{Fréquence de jeu et IMC.}
    \label{fig:freq_JA_IMC}
\end{figure}

#ECRIRE LE PARAGRAPHE



La même étude a été faite en ce qui concerne les jeux vidéos.

\begin{figure}[H]
    \centering
    \includegraphics[width=1\textwidth]{Images/P2/JV_IMC.png}
    \caption{Fréquence de jeu et IMC.}
    \label{fig:freq_JV_IMC}
\end{figure}
# ajouter commentaires sur ce graphique



\begin{figure}[H]
    \centering
    \includegraphics[width=1\textwidth]{Images/P2/JVfreqsimp_IMC.png}
    \caption{Fréquence de jeu et IMC.}
    \label{fig:freq_JV_IMC}
\end{figure}

#CHANGER LE PARAGRAPHE
Les individus en obésité comptent parmis eux plus d'un quart de joueurs
quotidiens (26,6\%), plus que toute autre catégorie d'IMC. Ce sont ensuite les
individus en insuffisance corporelle qui jouent le plus au quotidien (20,6\%
d'entre eux). Ce résultat est différent de celui qui concerne les jeux
d'argent, puisque les individus en insuffisance pondérale n'était pas plus
grands joueurs de jeux d'argent. Ils se démarquent en ce qui concerne les jeux
vidéos.




\newpage


\subsection{Comportement des joueurs vis à vis de leur santé}

Le comportement des joueurs vis à vis de leur santé est un élément d'étude
important, car au-delà d'un simple constat sur l'état de santé d'un individu,
il donne des habitudes de sa vie que l'on peut mettre en relation avec ses
habitudes de jeu.

\begin{figure}[H]
    \centering
    \includegraphics[width=1\textwidth]{Images/P2/JA&PS&sans_sport.png}
    \caption{Répartition des sportifs parmi les joueurs.}
    \label{fig:freq_joueur_JA_sport}
\end{figure}

Plus un individu fait du sport régulièrement, plus il joue aux jeux d'argent.
En effet, 34,9\% des sportifs quotidiens jouent à des jeux d'argent, contre
23,2\% de joueurs chez ceux qui ne pratiquent pas de sport. 

Ce résultat est à mettre en parallèle avec celui concernant seulement les
joueurs de paris sportifs, qui sont des jeux dont l'intérêt est lié au sport.

La différence entre un sportif régulier et un individu moins sportif est plus
prononcé sur ce graphique. 21,1\% des sportifs quotidiens jouent aux paris
sportifs, alors que seulement 5,6\% des non sportifs parient sur le sport. De
plus les sportifs quotidiens sont des joueurs réguliers de paris sportifs, car
6,8\% d'entre eux parient au moins une fois par semaine.

En ce qui concerne les liens entre la pratique de jeux vidéos et la pratique
sportive, ceux-ci sont moins évidents à établir.

\begin{figure}[H]
    \centering
    \includegraphics[width=1\textwidth]{Images/P2/}
    \caption{Répartition des sportifs parmi les joueurs.}
    \label{fig:freq_joueur_JV_sport}
\end{figure}

Les sportifs quotidiens sont de nouveau ceux qui jouent le plus aux jeux vidéos
(85,6\% d'entre eux jouent). A l'opposé, les non sportifs comptent moins de
joueurs parmis eux (76,5\%), mais une grande partie d'entre eux sont des
joueurs très réguliers: 20,3\% des non sportifs jouent presque tous les jours
aux jeux vidéo. C'est la plus grande part de joueurs très réguliers parmi les
quatre catégories de sportifs. Cette étude ne nous permet pas d'établir un lien
entre le comportement des individus vis à vis du sport et des jeux vidéos.\\

Il est également intéressant de comprendre le comportement des individus en ce
qui concerne la prise de rendez-vous chez le médecin, qui est un indicateur
d'un individu qui prend soin de sa santé ou non. 86,8\% des individus ayant
répondu au questionnaire sont allé chez le médecin dans l'année. Ce chiffre
varie si on étudie chaque catégorie de joueurs.

\begin{figure}[H]
    \centering
    \includegraphics[width=1\textwidth]{Images/Freq_JA_medecin.png}
    \caption{Part des individus ayant été chez le médecin au cours de l'année pour chaque catégorie de joueurs de jeux vidéo.}
    \label{fig:freq_medJV}
\end{figure}

Les plus gros joueurs de jeux d'argent sont allé moins souvent chez le médecin
que le reste de la population. 79,4\% y sont allé, ce qui est au moins 7 points
de moins que tout autre catégorie.

\begin{figure}[H]
    \centering
    \includegraphics[width=1\textwidth]{Images/Freq_JV_medecin.png}
    \caption{Part des individus ayant été chez le médecin au cours de l'année pour chaque catégorie de joueurs de jeux d'argent.}
    \label{fig:freq_medJA}
\end{figure}

Le même constat, mais moins prononcé, est visible en ce qui concerne les gros
joueurs de jeux vidéo. Cette fois-ci, on voit une légère augmentation de la
part de visites chez le médecin avec la diminution de la fréquence de jeux
vidéos. Les joueurs quotidiens se rapprochent cette fois du comportement du
reste de la population puisque 84,6\% d'entre eux se sont rendus chez le
médecin.