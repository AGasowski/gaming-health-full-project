\section{Qui sont ces joueurs ?}

\subsection{Quelles sont les pratiques de consommation du jeu ?}


\vskip 10pt Avant d'aborder les rapports observés entre la santé et la
consommation de jeu, il faut connaître les pratiques de jeu de la population
étudiée. En effet, il est important d’avoir une idée des différentes pratiques
des joueurs et des tendances qui se dégagent vis-à-vis du jeu dans cette
population, ainsi que leur profil socio-démographique. Sauf indication
contraire, toutes les informations présentées par la suite concernent les 12
mois précédant le remplissage du questionnaire. \vskip 10pt Intéressons-nous
d’abord aux consommateurs de jeu d’argent et à leurs préférences. Le jeu auquel
le plus de sondés ont joué est le ticket à gratter. En effet , 18,8\% des
sondés ont gratté au moins un ticket dans les 12 derniers mois. Ils sont bien
plus nombreux que ceux ayant joué au poker puisque ces derniers ne sont que
2,5\% des sondés. Mais cela ne veut pas dire que le ticket à gratter soit le
jeu d'argent le plus représentatif du quotidien des jeunes. En effet, si
relativement beaucoups de jeunes y ont joué, cela ne signifie pas pour autant
qu'ils y jouent régulierement. Cela ne vaut pas uniquement pour le ticket à
gratter. La consommation régulière de jeu d’argent ne concerne pas une grande
partie de la population.


\begin{figure}[H]
    \centering
    \includegraphics[width=1\textwidth]{Images/frequence_3_jeu.png}
    \caption{Fréquence des habitudes pour les paris sportifs, tickets à gratter et paris sportifs}
    \label{Fréquence des habitudes pour les paris sportifs, tickets à gratter et paris sportifs}
    \textit{Note de lecture: 0,6\% des intérrogés ne consomment des tickets à
    gratter qu'une fois par semaine.}\\
    \textit{La modalité "jamais" trop imporatante, a été enlevée du graphique
    par soucis de lisibilité}\\
    \textit{Source: Enquête ESCAPAD 2022.}

\end{figure}

\vskip10pt

\begin{figure}[H]
    \centering
    \includegraphics[width=1\textwidth]{Images/frequence_conso_JA.png}
    \caption{Répartition de la population selon sa consommation de jeux d'argent}
    \label{Répartition de la population selon sa consommation de jeux d'argent}
    \textit{Note de lecture: 19\% des intérrogés jouent à des jeux d'argent
    moins d'une fois par mois}\\
    \textit{Source: Enquête ESCAPAD 2022.}

\end{figure}

\vskip10pt

Le nombre de joueurs par catégorie est géneralement décroissant en fonction de
la fréquence de jeu. Ce sont seulement 3,9\% des sondés qui jouent plus d’une
fois par mois à des tickets à gratter. En revanche ce pourcentage s‘élève à 5,5
pour les joueurs de paris sportifs. Le pari sportif semble donc être un jeu
d’argent le plus consommé alors que le ticket à gratter est certes plus répendu
mais semble plus occasionnel. On peut alors penser que cette consommation de
paris sportifs s’explique par son accessibilité grâce au téléphone qui rends sa
pratique plus évidente. Or la part de personnes qui parient principalement sur
internet n’est pas si importante dans cet échantillon. Seulement 52\% des
parieurs sportif déclarent le faire majoritairement sur internet. D’après les
réponses du formulaire, les mises sur les paris sportifs peuvent représenter à
l’année un poste de consommation relativement important, notamment pour des
jeunes. La médiane des mises habituelles renseignées par les sondés est de 10€.
Les mises annuelles de certains représentent donc d’importantes sommes,
d’autant plus que plus un joueur pari, plus les sommes pariées sont en moyenne
importantes. De plus on peut poser l'hypothèse que les joueurs récurrents sont
plus enclins à miser plusieurs fois par jour.
\textit{il faut insérer ici un test du chi deux qui montre la dépendance entre les mises et la fréquence de jeu}


\begin{table}[H]
    \centering
    \captionsetup{justification=centering}
    \begin{tabular}{ |c|c|c|c| }
          \hline
           & Joueurs occasionnels & Joueurs hebdomadaires & Joueurs quotidiens
           \\ 
          \hline
          Moyenne des mises & 9,04€ & 10,29€ & 10,85€ \\     
    \end{tabular}
    \vskip 10pt \caption{Moyenne des mises habituelles selon la fréquence de
    jeu\\
    \textit{Source: Enquête ESCAPAD 2022.}} \label{Moyenne des mises
    habituelles selon la fréquence de jeu}
    \textit{Note de lecture: La moyenne des mises habituelles des joueurs quotidiens est de 10,85€}
\end{table}

Les jeux vidéo sont bien plus populaires que les jeux d’argent, d'abord parce
que plus de personnes y jouent. Cela s’explique peut-être par des normes
sociales, économiques et législatives, notamment pour cette tranche d’âge qui
n’est que partiellement majeure. 

\begin{figure}[H]
    \centering
    \includegraphics[width=1\textwidth]{Images/frequence_conso_JV.png}
    \caption{Répartition de la population selon sa consommation de jeux vidéo}
    \label{Répartition de la population selon sa consommation de jeux vidéo}
    \textit{Note de lecture: 17,5\% des intérrogés jouent à des jeux vidéo tous
    les jours ou presque.}\\
    \textit{Source: Enquête ESCAPAD 2022.}

\end{figure}

Seulement 19\% déclarent ne jamais jouer à des jeux vidéos alors que ce chiffre
s'élève à 73\% pour les jeux d’argent. 44\% des sondés jouent aux jeux vidéo au
moins plusieurs fois par semaine et 80\% au moins une fois par mois. De plus,
on joue plus ouvent aux jeux vidéos qu'aux jeux d'argent. En effet
contrairement aux jeux d'argent, le nombre de joueurs par catégorie n'est plus
décroissant en fonction de la fréquence de jeu. Plus de gens jouent à un jeu
video plusieurs fois par semaines (34\%) qu'une seule fois par semaine (12\%).
On peut même imaginer qu’on joue plus longtemps aux jeu vidéo qu’aux jeux
d’argent, mais il aurait fallu demander le temps passé à jouer à des jeux
d’argent pour pouvoir comparer. 

\newpage

\subsection{Quels sont les profils sociodémographiques des joueurs ?}


\vskip 10pt Apres une brève étude de la consommation des joueurs, il est
pertinent de regarder le profil de ces consommateurs. En effet si on étudie les
liens entre la santé et la consommation de tous les joueurs comme si leur seul
point commun était le jeu vidéo, alors cela serait nier les déterminants
sociologiques qui ont une grande importance sur la santé et sur la
consommation. \vskip 10pt On sait que tous les sondés sont de la même
génération. La variable qui peut expliquer en grande partie la consommation des
agents est le sexe. C’est une variable auquel beaucoup d’importance est
accordée, elle correspond à la troisième question du questionnaire. On s’attend
à ce que les hommes soient plus concernés par la consommation de jeu que les
femmes. Pour les jeux d'argent par exemple, l'étude \uline{The prevalence of
gambling and problematic gambling: a systematic review and meta-analysis} (Lucy
T TRAN et al, 2024) montre que 49,1\% des hommes ont joué à des jeux d'argent
dans l'année écoulée alors que ce n'est que le cas de 37,4\% des femmes. Des
observations similaires peuvent être faites dans notre enquête. Dans notre base
de données, c'est environ 33\% pour les hommes contre 23\% pour les femmes. Les
différences de résultats s'expliquent surement par la différence des
populations sondées, notament par leur âge et leur culture. En revanche, on
observe un ratio presque similaire (environ 1,4) du nombre de joueurs masculins
sur le nombre de joueuses de jeux d'argent. \vskip 10pt

\begin{center}\includegraphics[width=1\textwidth]{Images/freq_par_sexe.pdf}\\[2.0
cm]\end{center}
\textit{Note de lecture: 19\% des femmes jouent à des jeux vidéo moins d'une fois par mois}

\vskip 10pt




\newpage

\subsection{Profils-types des individus en fonction de leurs comportements et états de santé}

Pour développer cette analyse, nous nous appuyons sur l'ACM réalisée. L'analyse
de correspondances multiples est une méthode statistique qui nous permet de
dégager des profils-types d'individus dans notre étude. Pour cela nous avons
sélectionné les variables suivantes:
\begin{itemize}
    \item La fréquence de jeux d'argent (freqJA)
    \item La fréquence de jeux vidéo (freqJV)
    \item La fréquence de pratique sportive (freqSport)
    \item La catégorie ADRS (ADRS\_cat)
    \item La visite chez un médecin (freqMedecin)
\end{itemize}
Pour les trois premières variables, les modalités vont de 1 à 4 et sgnifient:
\begin{itemize}
    \item 1: "Jamais"
    \item 2: "Moins d'une fois par semaine"
    \item 3: "Au moins une fois par semaine"
    \item 4: "Au quotidien"
\end{itemize}
En ce qui concerne le score ADRS, celui-ci permet de quantifier le risque de
dépression chez un adolescent. Ici nous avons regroupé les individus en trois
groupes, 0 étant le groupe des individus sans risque de dépression, 1 avec un
risque modéré et 2 avec un risque élevé. Enfin pour la variable freqMedecin, 1
signifie que l'individu n'a pas vu de médecin au cours de l'année, et 2
signifie qu'il en a vu un.

\begin{figure}[H]
    \centering
    \includegraphics[width=1\textwidth]{P2/ACM.png}
    \caption{"Interactions entre pratiques de loisirs, santé et risques dépressifs chez les jeunes"}
    \label{fig: ACM}
\end{figure}

Tout d'abord, il faut rappeler que les axes de cette ACM ne résument qu'une
partie limitée de l'information (19,8\% au total). Il faut donc éviter la
surinterprétation de certains rapprochements.

\vspace{1em}

Essayons d'abord d'interprêter les dimensions. Commençons par l'analyse de la
dimension 1. Dans les valeurs négatives, nous retrouvons des individus très
actifs dans les jeux d'argent, le sport mais également, dans une moindre
mesure, dans les jeux vidéo. Dans les valeurs positives, nous avons plutôt des
individus plus sédentaires, peu engagés dans les activités de loisirs mais
également à risque psychologique élevé. La dimension 1 peut alors refléter un
gradient entre les profils “inactifs à risque dépressif” (à droite) et les
profils “très actifs” (à gauche).

\vspace{1em}

La dimension 2 est fortement expliquée par les individus qui jouent aux jeux
d'argent au quotidien, dans les valeurs positives. Cela montre un profil très
spécifique: jouer tous les jours à des jeux d’argent n’est pas anodin et
s’écarte clairement des autres groupes. De plus, ce comportement n'est pas
nécessairement corrélé à d'autres variables dans les deux premières dimensions.

\vspace{1em}

Des regroupements intéressants peuvent ressortir de ce graphe:
\begin{itemize}
    \item Autour de l'origine, nous retrouvons des profils modérés, avec des
    modalités intermédiaires.
    \item A droite, nous retrouvons des profils sédentaires, moins joueurs et à
    haut risque de dépression. Il semble donc que le risque de dépression chez
    l'individu soit lié à sa faible activité sportive ainsi que sa faible
    pratique de jeux vidéo. En d'autres termes, un individu en dépression est
    moins joueur et moins actif, et inversement.
    \item A gauche, nous retrouvons des individus à comportements intensifs, à
    la fois par rapport à leur pratique sportive mais également leur pratique
    de jeux d'argent. L'activité intensive d'un individu semble donc se
    répercuter sur plusieurs domaines à la fois.
    \item En ce qui concerne la visite chez un médecin, nous ne pouvons rien
    déduire de ce graphique.
\end{itemize}