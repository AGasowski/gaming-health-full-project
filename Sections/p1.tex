\section{Qui sont ces joueurs ?}

\subsection{Quelles sont les pratiques de consommation du jeu ?}


\vskip 10pt
Avant de s’intéresser aux rapports observés entre la santé et la consommation de jeu, il faut connaître les pratiques de jeu de la population étudiée. En effet, il est important d’avoir une idée des différentes pratiques des joueurs, les 
tendances qui se dégagent vis-à-vis du jeu dans cette population ainsi que leur profil 
socio-démographique. Sauf mention du contraire toute les informations présentées par la suite 
impliquent les 12 mois précédant le remplissage du questionnaire.
\vskip 10pt
Intéressons-nous d’abord aux consommateurs de jeu d’argent et à leurs préférences. Le jeu auquel 
le plus de sondés ont joué est le ticket à gratter. En effet , 19,1\% des sondés ont gratté au moins un ticket dans les 12 derniers mois. Ils sont bien plus nombreux que ceux ayant joué à une machine à sous puisque ces derniers ne sont que 2,7\% des sondés. Mais cela ne veut pas dire que le ticket à gratter soit 
le jeu d'argent qui occupe le plus de place dans le qotidien des jeunes. En effet, il est important de regarder la fréquence de ce 
jeu pour déterminer son importance dans la consommation des jeux d’argent. Les jeux d’argent ne 
concernent pas une grande partie de la population.


\begin{center}\includegraphics[width=1\textwidth]{Images/Frequence_conso_jeuxA.pdf}\\[2.0 cm]\end{center}
\textit{Note de lecture: 19\% des intérrogés jouent à des jeux d'argent moins d'une fois par mois}

\vskip10pt

Le nombre de joueurs par catégorie est clairement décroissant en fonction de la fréquence de jeu. Ce sont seulement 4,1\% des sondés qui jouent plus d’une fois par mois à des tickets à gratter. En 
revanche ce pourcentage s‘élève à 5,7 pour les joueurs de paris sportifs. Le pari sportif semble 
donc être le jeu d’argent le plus consommé alors que le ticket à gratter est certes plus répandu 
mais semble plus occasionnel. On peut alors penser que cette consommation de paris sportifs 
s’explique par son accessibilité grâce au téléphone qui rends sa pratique plus évidente. Or la part de personnes qui parient principalement sur internet n’est pas si 
importante dans cet échantillon. Seulement 52\% des parieurs sportif déclarent le faire 
majoritairement sur internet. D’après les réponses du formulaire, les mises sur les paris sportifs 
peuvent représenter à l’année un poste de consommation relativement important, notamment pour des 
jeunes. La médiane des mises habituelles renseignées par les sondés est de 10€. Les mises annuelles 
de certains représentent donc d’importantes sommes, d’autant plus que plus un joueur pari, plus les 
sommes pariées sont en moyenne importantes. De plus on peut poser l'hypothese que les joueurs 
récurents sont plus enclins à miser plusieurs fois par jour.
\textit{il faut insérer ici un test du chi deux qui montre la dépendance entre les mises et la fréquence de jeu}


\begin{table}[ht]
    \centering
    \captionsetup{justification=centering}
    \begin{tabular}{lccc}
          \hline
           & Joueurs occasionnels & Joueurs hebdomadaires & Joeuurs quotidiens \\ 
          \hline
          Moyenne des mises & 9,04€ & 10,29€ & 10,85€ \\     
    \end{tabular}
    \vskip 10pt
    \caption{Moyenne des mises habituelles selon la fréquence de jeu\\
    \textit{Source: Enquête ESCAPAD 2022.}}
    \textit{Note de lecture: La moyenne des mises habituelles des joueurs quotidiens est de 10,85€}
\end{table}

Les jeux vidéo sont bien plus populaires que les jeux d’argent, d'abord parce que plus de personnes y jouent. Cela s’explique peut-être par des normes sociales, économiques et 
législatives, notamment pour cette tranche d’âge qui n’est que partiellement majeure. 
\begin{center}\includegraphics[width=1\textwidth]{Images/Fréquence_conso_jeuV2.pdf}\\[2.0 cm]\end{center}
Seulement 19\% déclarent ne jamais jouer à des jeux vidéos alors que ce chiffre s'élève à 73\% pour les jeux d’argent. 44\% des sondés jouent aux jeux vidéo au moins plusieurs fois par semaine et 80\% au moins une fois par mois. De plus, on joue plus ouvent aux jeux vidéos qu'aux jeux d'argent. En effet contrairement aux jeux d'argent, le nombre de joueurs par catégorie n'est plus décroissant en fonction de la fréquence de jeu. Plus de gens jouent à un jeu video plusieurs fois par semaines (34\%) qu'une seule fois par semaine (12\%). 
On peut même imaginer qu’on joue plus longtemps aux jeu vidéo qu’aux jeux d’argent, mais il aurait fallu demander le temps passé à jouer à des jeux d’argent pour pouvoir comparer. 

\newpage

\subsection{Quels sont les profils sociodémographiques des joueurs ?}


\vskip 10pt
Apres une brève étude de la consommation des joueurs, il est pertinent de regarder le profil de ces 
consommateurs. En effet si on étudie les liens entre la santé et la consommation de tous les joueurs 
comme si leur seul point commun était le jeu vidéo, alors cela serait nier les déterminants 
sociologiques qui ont une grande importance sur la santé et sur la consommation.
\vskip 10pt
On sait que tous les sondés sont de la même génération. La variable qui peut expliquer en grande partie la consommation des agents est le sexe. C’est une variable auquel beaucoup d’importance est accordée, 
elle correspond à la troisième question du questionnaire. On s’attend à ce que les hommes soient plus 
concernés par la consommation de jeu que les femmes. Pour les jeux d'argent par exemple, l'étude \underline{The prevalence of gambling and problematic gambling: a systematic review and meta-analysis} (Lucy T TRAN et al, 2024) montre que 49,1\% des hommes ont joué à des jeux d'argent dans l'année écoulée alors que ce n'est que le cas de 37,4\% des femmes.
Des observations similaires peuvent être faites dans notre enquête. Dans notre base de données, c'est environ 33\% pour les hommes contre 23\% pour les femmes. Les différences de résultats s'expliquent surement par la différence des populations sondées, notament par leur âge et leur culture. En revanche, on observe un ratio presque similaire (environ 1,4) du nombre de joueurs masculins sur le nombre de joueuses de jeux d'argent. 
\vskip 10pt

\begin{center}\includegraphics[width=1\textwidth]{Images/freq_par_sexe.pdf}\\[2.0 cm]\end{center}
\textit{Note de lecture: 19\% des femmes jouent à des jeux vidéo moins d'une fois par mois}

\vskip 10pt




\vskip 10pt