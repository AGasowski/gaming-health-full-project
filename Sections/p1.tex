\section{Une brève déscription de la population de joueurs}

\subsection{Quelles sont les pratiques de consommation du jeu ?}


\vskip 10pt
Avant de s’intéresser aux rapports observés entre la santé et la consommation de jeu, il est important d’ériger un bref portrait de la catégorie des sondés qui nous intéresse, celle des joueurs. En effet, il est important d’avoir une idée des différentes pratiques des joueurs, les tendances qui se dégagent vis-à-vis du jeu dans cette population ainsi que leur profil socio-démographique. Sauf mention du contraire toute les informations présentées par la suite impliquent les 12 mois précédant le remplissage du questionnaire.

	Intéressons-nous d’abord aux consommateurs de jeu d’argent et à leurs préférences. Le jeu auquel le plus de sondés ont joué est le jeu de grattage. En effet au cours des 12 mois précédant le remplissage du questionnaire, 19,1% des sondés ont gratté au moins un ticket dans l’espoir qu’il soit gagnant. Ils sont bien plus nombreux que ceux ayant joué à une machine à sous puisque ces derniers ne sont que 2,7% des sondés. Mais cela ne veut pas dire que les tickets à gratter soient le plus démocratisé des jeux d’argents. En effet, il est important de regarder la fréquence de ce jeu pour déterminer son importance dans la consommation des jeux d’argent. Les jeux d’argent ne concernent pas une grande partie de la population.

\begin{figure}[htb]
    \captionsetup{justification=centering}
    \begin{minipage}[t]{0.49\textwidth}
        \centering
        \includegraphics[scale = 1.0]{Images/Frequence_conso_jeuxA.pdf}
    \end{minipage}
    \hfill
    
    \textit{Note de lecture: 19 pourcents des intérrogés jouent à des jeux d'argent moins d'une foi par mois}
\end{figure}


\newpage

\subsection{Quels sont les profils sociodémographiques des joueurs ?}


\vskip 10pt




\newpage


\subsection{Troisième sous-partie}
