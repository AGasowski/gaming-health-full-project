\section{Une brève déscription de la population de joueurs}

\subsection{Quelles sont les pratiques de consommation du jeu ?}


\vskip 10pt
Avant de s’intéresser aux rapports observés entre la santé et la consommation de jeu, il est 
important d’ériger un bref portrait de la catégorie des sondés qui nous intéresse, celle des 
joueurs. En effet, il est important d’avoir une idée des différentes pratiques des joueurs, les 
tendances qui se dégagent vis-à-vis du jeu dans cette population ainsi que leur profil 
socio-démographique. Sauf mention du contraire toute les informations présentées par la suite 
impliquent les 12 mois précédant le remplissage du questionnaire.
\vskip 10pt
Intéressons-nous d’abord aux consommateurs de jeu d’argent et à leurs préférences. Le jeu auquel 
le plus de sondés ont joué est le jeu de grattage. En effet au cours des 12 mois précédant le 
remplissage du questionnaire, 19,1\% des sondés ont gratté au moins un ticket dans l’espoir qu’il 
soit gagnant. Ils sont bien plus nombreux que ceux ayant joué à une machine à sous puisque ces 
derniers ne sont que 2,7\% des sondés. Mais cela ne veut pas dire que les tickets à gratter soient 
le plus démocratisé des jeux d’argents. En effet, il est important de regarder la fréquence de ce 
jeu pour déterminer son importance dans la consommation des jeux d’argent. Les jeux d’argent ne 
concernent pas une grande partie de la population.


\begin{center}\includegraphics[width=1\textwidth]{Images/Frequence_conso_jeuxA.pdf}\\[2.0 cm]\end{center}
\textit{Note de lecture: 19\% des intérrogés jouent à des jeux d'argent moins d'une fois par mois}

\vskip10pt

Ce sont seulement 4,1\% des sondés qui jouent plus d’une fois par mois à des tickets à gratter. En 
revanche ce pourcentage s‘élève à 5,7 pour les joueurs de paris sportifs. Le pari sportif semble 
donc être le jeu d’argent le plus consommé alors que le ticket à gratter est certes plus répandu 
mais semble plus occasionnel. On peut alors penser que cette consommation de paris sportifs 
s’explique par le fait qu’elle soit accessible directement sur le téléphone et soit donc moins 
contraignante. Or la part de personnes qui parient principalement sur internet n’est pas si 
importante dans cet échantillon. Seulement 52\% des parieurs sportif déclarent le faire 
majoritairement sur internet. D’après les réponses du formulaire, les mises sur les paris sportifs 
peuvent représenter à l’année un poste de consommation relativement important, notamment pour des 
jeunes. La médiane des mises habituelles renseignées par les sondés est de 10€. Les mises annuelles 

de certains représentent donc d’importantes sommes, d’autant plus que plus un joueur pari, plus les 
sommes pariées sont en moyenne importantes. De plus on peut poser l'hypothese que les joueurs 
récurents sont plus enclins à miser plusieurs fois par jour.

\begin{table}[ht]
    \centering
    \captionsetup{justification=centering}
    \begin{tabular}{lccc}
          \hline
           & Joueurs occasionnels & Joueurs hebdomadaires & Joeuurs quotidiens \\ 
          \hline
          Moyenne des mises & 9,04€ & 10,29€ & 10,85€ \\     
    \end{tabular}
    \vskip 10pt
    \caption{Répartition de l'état de santé\\
    \textit{Source: Enquête ESCAPAD 2022.}}
    \textit{Note de lecture: La moyenne des mises habituelles des joueurs quotidiens est de 10,85€}
\end{table}

Les jeux vidéo sont bien plus populaires que les jeux d’argent : plus de personnes y joue et on y 
joue plus souvent. Cela s’explique d’abord par des préférences personnelles, motivées oui ou non 
par une rationalité économique. Mais cela s’explique peut-être aussi par des normes sociales et 
législatives, notamment pour cette tranche d’âge qui n’est que partiellement majeure. On peut même 
imaginer qu’on joue plus longtemps aux jeu vidéo qu’aux jeux d’argent, mais il aurait fallu 
demander le temps passé à jouer à des jeux d’argent pour pouvoir comparer.44\% des sondés jouent 
au moins plusieurs fois par semaine. 

\begin{center}\includegraphics[width=1\textwidth]{Images/Fréquence_conso_jeuV2.pdf}\\[2.0 cm]\end{center}

\newpage

\subsection{Quels sont les profils sociodémographiques des joueurs ?}


<<<<<<< HEAD
=======
\vskip 10pt
Apres une brève étude de la consommation des joueurs, il est important de regarder le profil de ces 
consommateurs. En effet si on étudie les liens entre la santé et la consommation de tous les joueurs 
comme si leur seul point commun était le jeu vidéo, alors cela serait nier les déterminants 
sociologiques qui ont une grande importance sur la santé et sur la consommation.
\vskip 10pt
On sait que tous les sondés sont de la même génération. La variable qui peut expliquer en grande partie 
la consommation des agents est le sexe. C’est une variable auquel beaucoup d’importance est accordée, 
elle correspond à la troisième question du questionnaire. On s’attend à ce que les hommes soient plus 
concernés par la consommation de jeu que les femmes. L’étude de l’enquête certifie largement cette 
hypothèse dans les deux types de jeu.
\vskip 10pt

\begin{figure}[htb]
    \captionsetup{justification=centering}
    \begin{minipage}[t]{0.49\textwidth}
        \centering
        \includegraphics[scale = 0.70]{Images/freq_par_sexe.pdf}
    \end{minipage}
    \hfill
    \textit{Note de lecture: 19\% des femmes jouent à des jeux vidéo moins d'une foi par mois}
\vskip 10pt
Une autre variable socio-démographique est la catégorie socio-professionnelle des parents.
Elle est d’autant plus importante ici car les sujets sont jeunes et beaucoup dépendent financièrement,
au moins partiellement, de leurs parents. Intéressons-nous ici à la PCS du père.
Lorsqu’on analyse cette variable avec la consommation de jeu d’argent on se rends compte qu’elle 
semble déterminer à grande échelle cette pratique. Nous avons évoqué plus haut l’aspect 
onéreux des jeux d’argent, et cela peut donc 
rejoindre le fait que les enfants de cadre soient ceux qui jouent le plus souvent.

\begin{figure}[htb]
    \captionsetup{justification=centering}
    \begin{minipage}[t]{0.49\textwidth}
        \centering
        \includegraphics[scale = 0.70]{Images/tableau_jeuA_PCSpere.pdf}
    \end{minipage}
    \hfill
    \textit{Note de lecture: 18,6\% des individus avec un père ouvrier jouent à des jeux d'argent moins d'une foi par mois}
\vskip 10pt
>>>>>>> 03fd0b955edf560890ee9028006685254689ac47


