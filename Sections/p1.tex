\section{Qui sont ces joueurs ?}

\subsection{Des jeux plus populaires que d'autres}


\vskip 10pt

Il est important d’avoir une idée des différentes pratiques des joueurs et des
tendances qui se dégagent vis-à-vis du jeu dans cette population, ainsi que
leur profil socio-démographique.

\vskip 10pt

Intéressons-nous d’abord aux consommateurs de jeu d’argent et à leurs
préférences.

\begin{figure}[H]
    \centering
    \includegraphics[width=1\textwidth]{Images/joueur_3JA.png}
    \caption{Consommation de paris sportifs, poker et tickets à gratter}
    \label{joueurs 3JA}
    \textit{Note de lecture: 97,5\% des interrogés n'ont pas joué au poker dans
    les 12 derniers mois.}\\
    \textit{Source: Enquête ESCAPAD 2022.}

\end{figure}

C'est au ticket à gratter que le plus d'individus se sont essayé. En effet,
18,9\% d'entre eux en ont gratté un dans les 12 derniers mois. Ils sont bien
plus nombreux que ceux ayant joué au poker puisque ces derniers ne sont que
2,5\% des sondés (Figure \ref{joueurs 3JA}). Mais cela ne veut pas dire que le
ticket à gratter soit le jeu d'argent le plus représentatif du quotidien des
jeunes. En effet, si beaucoup de jeunes y ont joué, cela ne signifie pas pour
autant qu'ils y jouent régulièrement. Par exemple, seulement 5,9\% (5,5\% +
0,4\%) d'entre eux y jouent toutes les semaines ou tous les jours (Figure
\ref{frequence 3JA}). Le ticket à gratter a certes plus d'amateurs, mais semble
plus occasionnel. Par contre les joueurs de poker ou de paris sportifs jouent
plus souvent. Pour le poker, ils sont 24,1\% à y jouer toutes les semaines ou
tous les jours et ce chiffre s'élève à 25,8\% pour les parieurs sportifs
(Figure \ref{frequence 3JA}). On peut penser que ces consommations de paris
sportifs et de poker s’expliquent par l'accessibilité due au téléphone qui rend
leur pratique plus évidente. Or la part de personnes qui parient principalement
sur internet n’est pas si importante. Seulement 52\% des parieurs sportifs
déclarent le faire majoritairement sur internet d'après l'enquête ESCAPAD 2022.
Mais d'une manière générale, la consommation régulière de jeu d’argent ne
concerne pas une grande partie de la population. 73\% de la population
n'y a pas joué de l'année (Figure \ref{frequence JA}).


\begin{figure}[H]
    \centering
    \includegraphics[width=1\textwidth]{Images/frequence_3JA.png}
    \caption{Fréquence de jeu pour les joueurs de poker, paris sportifs et tickets à gratter}
    \label{frequence 3JA}
    \textit{Note de lecture: 22,8\% des consommateurs de paris sportifs y
    jouent toutes les semaines.}\\
    \textit{Source: Enquête ESCAPAD 2022.}

\end{figure}

\vskip10pt

\begin{figure}[H]
    \centering
    \includegraphics[width=1\textwidth]{Images/frequence_conso_JA.png}
    \caption{Répartition de la population selon sa consommation de jeux d'argent}
    \label{frequence JA}
    \textit{Note de lecture: 19\% des interrogés jouent à des jeux d'argent
    moins d'une fois par mois}\\
    \textit{Source: Enquête ESCAPAD 2022.}

\end{figure}

\vskip10pt

Au-delà la fréquence de jeu, on peut percevoir l'intensité de la consommation
de jeux à travers les mises pariées. La médiane des mises habituelles
renseignées par les sondés est de 10€. Les mises annuelles de certains
représentent donc d’importantes sommes pour les joueurs réguliers. C'est
d’autant plus le cas puisque plus un joueur parie, plus les sommes pariées sont
en moyenne importantes (Tableau \ref{mises paris}). Un test du
khi\textsuperscript{2} ($p-value = 8,8 \times 10^{-12}$) nous confirme qu'il y
a effectivement un lien statistique entre la fréquence de jeux d'argent et la
mise habituelle (voir \hyperref[ann:correlation]{Annexe C}). Cependant, le
calcul du V de Cramer nous indique que l'intensité de cette association est
faible (V = 0,177).

Le graphique des résidus (Figure \ref{Résidus mises}) nous permet d'affiner
l'interprétation de l'association. En effet, on observe que les joueurs
occasionnels sont significativement plus nombreux que prévu à miser des faibles
montants (0-10€). Par ailleurs, les joueurs quotidiens se distinguent avec une
proportion notable de mises très élevées (+50€), dépassant largement les
attentes théoriques.

\vskip 10pt

\begin{figure}[H]
    \centering
    \includegraphics[width=1\textwidth]{Images/Residus/res_JA_mise.png}
    \caption{Tableau des résidus standardisés}
    \label{Résidus mises}
    \textit{Note de lecture: La catégorie des joueurs quotidiens misant plus de
    50 € est surreprésentée, alors que celle des individus jouant moins d'une
    fois par semaine est sous-représentée.}\\
    \textit{La règle choisie pour analyser les valeurs des résidus est énoncée
    en \hyperref[ann:residus_IMC]{Annexe D}}\\
    \textit{Source: Enquête ESCAPAD 2022.}

\end{figure}


\begin{table}[H]
    \centering
    \captionsetup{justification=centering}
    \begin{tabular}{ |c|c|c|c| }
          \hline
           & Joueurs occasionnels & Joueurs hebdomadaires & Joueurs quotidiens
           \\
          \hline
          Moyenne des mises & 9,04€ & 10,29€ & 10,85€ \\
    \end{tabular}
    \caption{Moyenne des mises habituelles selon la fréquence de jeu}
    \label{mises paris}
    \vskip 10pt \textit{Note de lecture: La moyenne des mises habituelles des
    joueurs quotidiens est de 10,85€} \\
    \textit{Source: Enquête ESCAPAD 2022.}
\end{table}
\vskip10pt

Les jeux vidéo sont bien plus populaires que les jeux d’argent. Cela s’explique
peut-être par des normes sociales, économiques et législatives, notamment pour
cette tranche d’âge qui n’est que partiellement majeure.

\begin{figure}[H]
    \centering
    \includegraphics[width=1\textwidth]{Images/frequence_conso_JV.png}
    \caption{Répartition de la population selon sa consommation de jeux vidéo}
    \label{freq jv}
    \textit{Note de lecture: 17,5\% des intérrogés jouent à des jeux vidéo tous
    les jours ou presque.}\\
    \textit{Source: Enquête ESCAPAD 2022.}

\end{figure}

Seulement 19\% déclarent ne jamais jouer à des jeux vidéo (Figure \ref{freq jv}) alors que ce chiffre
s'élève à 73\% pour les jeux d’argent (Figure \ref{frequence JA}). 55\% des sondés jouent aux jeux vidéo au
moins plusieurs fois par semaine et 80\% au moins une fois par mois (Figure \ref{freq jv}). En plus 
d'être plus nombreux à jouer aux jeu vidéo,
les joueurs jouent plus souvent aux jeux vidéo qu'aux jeux d'argent. En effet,
contrairement aux jeux d'argent, le nombre de joueurs par catégorie n'est pas
décroissant en fonction de la fréquence de jeu. Plus de gens jouent à un jeu
vidéo plusieurs fois par semaine (34\%) qu'une seule fois par semaine (12\%) (Figure \ref{freq jv}).
On peut même imaginer qu’on joue plus longtemps aux jeux vidéo qu’aux jeux
d’argent, mais il aurait fallu demander le temps passé à jouer à des jeux
d’argent pour pouvoir comparer.


\subsection{Des consommations genrées}


\vskip 10pt

Apres une brève étude de la consommation des joueurs, il est pertinent de
regarder le profil de ces consommateurs. En effet, si on étudie les liens entre
la santé et la consommation de tous les joueurs comme si leur seul point commun
était le jeu vidéo, alors cela serait nier les déterminants sociologiques qui
ont une grande importance sur la santé et sur la consommation. \vskip 10pt Une
variable qui différencie largement les pratiques culturelles est le genre. On
s’attend à ce que les hommes soient plus concernés par la consommation de jeu
que les femmes. Cette attente est confirmée par la littérature. 49,1\% des
hommes ont joué à des jeux d'argent dans l'année écoulée, alors que ce n'est que
le cas de 37,4\% des femmes (Lucy T TRAN et al, 2024 \cite{gambling}). Dans
notre base de données, c'est 32,7\% des hommes qui y ont joué contre 22,6\% des
femmes (Figure \ref{freq_par_genre}). On constate une différence entre les deux
enquêtes en termes de point de pourcentage. Cependant, on peut noter que les
deux ratios de joueurs par rapport au joueuses sont presque similaires. Celui
de l'enquête Escapade est de 1,2 alors que le premier est de 1,3. \\
La pratique du jeu vidéo est encore plus genrée que celle des jeux d'argent.
Les jeux vidéo ont d'abord été popularisés dans des milieux masculins, mais on
peut observer une féminisation de ce secteur (Samuel Coavoux, 2019
\cite{gamer}). En 2011, 29,5\% des garçons de 17 ans jouent tous les jours aux
jeux vidéo, alors que c'est seulement 3\% pour les filles du même âge
(Berthomier et Octobre, 2011 \cite{enfance_loisirs}). Comparer ces résultats
avec ceux de l'enquête ESCAPAD 2022 semble indiquer que la fréquence de jeu
vidéo est moins genrée qu'il y a quelques années. En 2022, 26,1\% des hommes
jouent tous les jours contre 8,7\% des femmes (Figure \ref{freq_par_genre}). En
revanche, même si la fréquence de jeu vidéo s'homogénise selon le genre, cela
ne signifie pas qu'elles se ressemblent de plus en plus. Les femmes ne jouent
pas aux mêmes jeux que les hommes qui ont une pratique plus intensive (Samuel
Coavoux, 2019 \cite{gamer}).

\vskip 10pt

Ce n'est pas seulement que les hommes jouent plus souvent aux jeux vidéo que
les femmes, mais on observe même une tendance opposée selon le genre. Pour les
hommes, il y a presque une relation positive entre fréquence de jeu vidéo et
nombre de joueurs, alors que cette relation est presque négative pour les
femmes (Figure \ref{freq_par_genre}).

\vskip 10pt

\begin{figure}[H]
    \centering
    \includegraphics[width=1\textwidth]{Images/freq_par_sexe.png}
    \caption{Fréquence de jeu par genre}
    \label{freq_par_genre}
    \textit{Note de lecture: 19\% des femmes jouent à des jeux vidéo moins d'une fois par mois}
    \textit{Source: Enquête ESCAPAD 2022.}
\end{figure}


\vskip 10pt

En plus du sexe, d'autres variables sociodémographiques peuvent caractériser
les joueurs. Ces variables peuvent aussi expliquer l'état de santé d'un
individu. Certaines régions, les déserts médicaux, ne favorisent pas la
consultation d'un médecin. La classe sociale détermine des pratiques qui ont
des conséquences sur la santé. La base de données donne la  PCS (profession et
catégorie sociale) des parents des individus. Le profil des joueurs selon leur
origine sociale peut donc expliquer en partie les corrélations entre jeux et
santé. (voir \hyperref[ann:pcs]{Annexe A})





\subsection{Profiler les joueurs selon leur comportement}

Pour développer cette analyse, nous nous appuyons sur l'ACM réalisée. L'analyse
de correspondances multiples est une méthode statistique qui nous permet de
dégager des profils-types d'individus dans notre étude. Pour cela nous avons
sélectionné les variables suivantes:
\begin{itemize}
    \item La fréquence de jeux d'argent (freqJA)
    \item La fréquence de jeux vidéo (freqJV)
    \item La fréquence de pratique sportive (freqSport)
    \item La catégorie ADRS (ADRS\_cat)
    \item La visite chez un médecin (freqMedecin)
\end{itemize}
Pour les trois premières variables, les modalités vont de 1 à 4 et signifient:
\begin{itemize}
    \item 1: "Jamais"
    \item 2: "Moins d'une fois par semaine"
    \item 3: "Au moins une fois par semaine"
    \item 4: "Au quotidien"
\end{itemize}
En ce qui concerne le score ADRS, celui-ci permet de quantifier le risque de
dépression chez un adolescent. Ici nous avons regroupé les individus en trois
groupes, 0 étant le groupe des individus sans risque de dépression, 1 avec un
risque modéré et 2 avec un risque élevé. Enfin pour la variable freqMedecin, 1
signifie que l'individu n'a pas vu de médecin au cours de l'année, et 2
signifie qu'il en a vu un.

\begin{figure}[H]
    \centering
    \includegraphics[width=1\textwidth]{P2/ACM.png}
    \caption{"Interactions entre pratiques de loisirs, santé et risques dépressifs chez les jeunes"}
    \label{fig: ACM}
\end{figure}

Tout d'abord, il faut rappeler que les axes de cette ACM ne résument qu'une
partie limitée de l'information (19,8\% au total). Il faut donc éviter la
surinterprétation de certains rapprochements. Nous n'avons analysé que les deux
premières dimensions, en adéquation avec le critère du coude
(\hyperref[ann:coude1]{Annexe E}).

\vspace{1em}

Essayons d'abord d'interpréter les dimensions. Commençons par l'analyse de la
dimension 1. Dans les valeurs négatives, nous retrouvons des individus très
actifs dans les jeux d'argent, le sport mais également, dans une moindre
mesure, dans les jeux vidéo. Dans les valeurs positives, nous avons plutôt des
individus plus sédentaires, peu engagés dans les activités de loisirs mais
également à risque psychologique élevé. La dimension 1 peut alors refléter un
gradient entre les profils “inactifs à risque dépressif” (à droite) et les
profils “très actifs” (à gauche).

\vspace{1em}

La dimension 2 est fortement expliquée par les individus qui jouent aux jeux
d'argent au quotidien, dans les valeurs positives. Cela montre un profil très
spécifique: jouer tous les jours à des jeux d’argent n’est pas anodin et
s’écarte clairement des autres groupes. De plus, ce comportement n'est pas
nécessairement corrélé à d'autres variables dans les deux premières dimensions.

\vspace{1em}

Des regroupements intéressants peuvent ressortir de ce graphe:
\begin{itemize}
    \item Autour de l'origine, nous retrouvons des profils modérés, avec des
    modalités intermédiaires.
    \item A droite, nous retrouvons des profils sédentaires, et à haut risque
    de dépression. Il semble donc que le risque de dépression chez l'individu
    soit lié à sa faible activité sportive. De plus, les individus à risque de
    dépression modéré sont proches des individus peu sportifs et ne jouant pas
    aux jeux vidéo. En d'autres termes, un individu à risque de dépression est
    moins joueur et moins actif, et inversement.
    \item A gauche, nous retrouvons des individus à comportements intensifs, à
    la fois par rapport à leur pratique sportive mais également leur pratique
    de jeux d'argent. L'activité intensive d'un individu semble donc se
    répercuter sur plusieurs domaines à la fois.
    \item En ce qui concerne la visite chez un médecin, nous ne pouvons rien
    déduire de ce graphique.
\end{itemize}
