\section*{Conclusion}
\addcontentsline{toc}{section}{Conclusion}
Les jeux d’argents et les jeux vidéo ont un réel impact sur la santé tant mentale que physique. 
Si ces jeux peuvent être une source de divertissement ou de revenu pour le consommateur, ils 
comportent aussi des effets néfastes sur la santé des joueurs addicts.
L’IMC, un indicateur évaluant la corpulence d’une personne, fait clairement le lien entre la 
santé physique d’une personne et sont addiction. Les personnes avec la corpulence la plus élevée 
sont en effet ceux qui consomment le plus de jeux. De même l'étude a montré que les jeux ont un 
impact sur la santé mentale des individus. Des troubles psychologiques comme un état important de 
stress, des formes de dépressions ou simplement un mal-être concernent une partie des joueurs 
dépendants.
Néanmoins il ne faut pas réduire les problèmes de santé à l’addiction aux jeux d’argent ou aux 
jeux vidéo. D’autres facteurs ne sont pas pris en compte dans l’étude et affectent grandement la 
santé morale et physique comme la situation familiale et professionnelle, l’alimentation ou la 
condition de vie par exemple. Ceux-ci peuvent nuire directement à la santé d’un individu ou bien 
plonger ce même individu plus facilement dans une addiction quelle qu’elle soit.
Enfin l’impact du marché du jeu vidéo et du jeu d’argent pose la question sur la régulation de 
ces marchés, sur le fait de sensibiliser, de changer les manières d’éduquer les enfants pour les 
préparer aux risques des jeux. L’encadrement de ces industries et la prévention des comportements 
à risque restent des enjeux majeurs dans le futur.
