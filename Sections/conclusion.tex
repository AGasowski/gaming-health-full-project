\section*{Conclusion}
\addcontentsline{toc}{section}{Conclusion}


Nous nous étions demandé : Dans quelle mesure existe-il une relation entre les
comportements addictifs liés aux jeux et la santé?


Contrairement à certains a priori, les liens entre la 
consommation de jeux et la santé ne sont pas si évidents. Il existe 
bien quelques corrélations entre des pratiques de jeu et 
la santé des joueurs. Cependant ces corrélations peuvent 
parfois s’avérer surprenantes.


C’est notamment le cas de la pratique d’activité physique. 
Malgré le cliché du gamer cloîtré chez lui, on observe que 
les jeunes qui jouent très souvent aux jeux vidéo ont 
tendance à faire plus régulièrement du sport que ceux qui y 
jouent moins. On peut faire une observation similaire avec 
les parieurs sportifs. Bien sûr, jouer à intervalles 
réguliers n’encourage pas pour autant la pratique de sport.
Mais d’autres relations entre consommation de jeu et santé 
ne sont pas de si positives pour les joueurs. C’est par 
exemple le cas de l’IMC par rapport à la pratique de sport. 
Un joueur de jeu vidéo quotidien a plus de chance d’être 
obèse ou en insuffisance pondérale qu’un joueur moins 
régulier. On peut aussi observer qu'un individu ne jouant pas 
aux jeux vidéo présente moins de risques d'épisode dépressif caractérisé.


La consommation de jeu n'est pas à considérer comme determinant de la santé.
D’autres facteurs ne sont pas pris en compte dans
l’étude et affectent grandement la santé tel que la situation
familiale et professionnelle, l’alimentation ou la condition de vie par
exemple. Ces derniers peuvent nuire directement à la santé d’un individu ou bien
au contraire favoriser le bien-être.


Les résultats de ce rapport sont à remettre en perspective de part la
nature de la base de données. Elle ne conçerne qu'une génération avec un intervalle de 3 ans seulement.
D'autant plus que la consommation de jeux video et jeux d'argent est une pratique très différente
d'une génération à une autre. On peut aussi supposer que les résulatas auraient été très 
différents et que des questions plus précises sur la santé auraient ouvert de nouvelles portes