\section*{Conclusion}
\addcontentsline{toc}{section}{Conclusion}

Contrairement à certains a priori, les liens entre la consommation de jeux et
la santé ne sont pas si évidents. Il existe effectivement quelques corrélations
entre des pratiques de jeu et la santé des joueurs, mais ces corrélations
peuvent parfois s’avérer surprenantes.


C’est notamment le cas de la pratique d’activités physiques. Malgré certains
stéréotypes, on observe que les jeunes qui jouent très souvent aux jeux vidéo
ont tendance à faire plus régulièrement du sport que ceux qui y jouent moins.
On observe un phénomène encore plus marqué chez les parieurs sportifs. Mais
toutes les relations entre jeux et santé ne sont pas si enthousiastes pour les
joueurs. C’est par exemple le cas de l’IMC par rapport à la pratique du jeu.
Les joueurs quotidiens de jeux vidéo  sont surreprésentés parmi les joueurs
obèses ou en insuffisance pondérale. On peut aussi observer qu'un individu ne
jouant pas aux jeux vidéo présente moins de risques d'épisode dépressif
caractérisé.


La consommation de jeu n'est pas à considérer comme déterminant de la santé, et
même si des relations sont observées cela ne prouve en rien une quelconque
causalité. D’autres facteurs ne sont pas pris en compte dans l’étude et
affectent grandement la santé tels la situation familiale et professionnelle,
l’alimentation ou les conditions de vie par exemple. Ces derniers peuvent nuire
directement à la santé d’un individu ou bien au contraire favoriser le
bien-être.


Les résultats de ce rapport sont à remettre en perspective du fait de la nature
de la base de données. Elle ne concerne qu'un petit échantillon d'individus du
même âge, alors que la consommation de jeux est une pratique très différente
d'une génération à l'autre. On peut aussi supposer que les résultats auraient
été très différents et que des questions plus précises sur la santé auraient
ouvert de nouvelles portes. Il aurait été intéressant de connaître les
symptômes les plus récurrents chez chacun (maux de ventre, insomnie, etc.) et
d'ainsi pouvoir les étudier.


Les jeux vidéo et les jeux d'argent sont des pratiques relativement modernes.
Il serait pertinent de croiser cette étude avec une autre, étudiant le lien
entre la santé et les pratiques numériques en général.